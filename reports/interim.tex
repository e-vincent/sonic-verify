\documentclass[11pt]{scrartcl}
\usepackage{a4, fullpage}

\setlength{\parskip}{0.3cm}
\setlength{\parindent}{0cm}

\begin{document}

\title{Musical Concurrent Programming \\ With Sonic Pi}
\subtitle{Interim Report}
\author{Eleanor Vincent}
\date{\today}
\maketitle

\section{Introduction} % <- Motivation: Project Outline

\subsection{Motivation}
For many years it has not been a requirement that children should learn much
about the vast field of computing during their formative years in UK Education.
Some of the earliest significant pieces of work towards the education of
computing started with the invention of Logo, an adaptation of the LISP 
language, most remembered for its use of ``turle graphics''. Some time after 
this came the Computers in the Cirriculum Project, funded from 1972 to 1991 by 
the Schools Council and subsequently by the Microelectronics Education 
Programme in 1981. The first microcomputers appeared in both UK Primary and 
Secondary Schools in 1979. The Commodore Pets aided both spelling and 
arthimetic practice as well as the ability to teach either BASIC or Logo. The 
1980's saw a huge amount of legislative reform and technical efforts in the 
vein of giving young people the ability to work with computers during their 
formative years but over the course of the late 80's and early 90's this focus 
on programming ability gives way to simply the education of practical use of 
existing computer software instead of a focus on the fundamentals behind these 
applications \cite{naec}.

In 2013, Ofsted published a report documenting their findings relating to ICT 
in UK schools from 2008 to 2011 and found that in half of all secondary 
schools, school leavers had not been given adequete education to move into a 
technical career in their future. In 2007, 81,100 pupils were enrolled in the 
ICT GCSE but this had fallen to 31,800 pupils by 2011 \cite{DfEO13} with a 
notable lack in the education of key skills such as computer programming 
itself. This lack was found to be as much a lack in knowledge from the 
teacher's as much as the cirriculum's failure to address the issues.

In 2012, the Government began to recog the signifiance of Computer Science and 
has replaced the National ICT curriculum with a revised Computing cirruculum. 
The new Computing Program of Study \cite{DfE13} aims to enable pupils to 
understand the world of computing, giving them the ability to think logically 
and apply the fundamental principles of the discpline to their real-world 
environments.

Along a similar vein there is a general struggle within the humanties courses 
available in schools to remain relevant in light of a quickly developing 
technical world. Music schemes within the UK frequently report to have 
suffered funding cuts and education is focused on learning a variety of 
specific instruments, with little focus on musical technology until the later 
years of education.

As well as the necessity of Computer Science and Music within schools, there 
is an increasing recognition of the power of programming amongst the general 
populace. A growing hacker and maker movement has been making programming a 
much more accessible skill \cite{BAD14}; it presents itself as a viable and 
useful hobby amongst a vast range of ages and professions and this is as much 
because of the availability of useful resources that would be frequently used 
in such situations as a school classroom. There is existing research that has 
gone into the viability of programming tools for professional artists, as 
reported at PPIG \cite{Ch12,BC05}, and investigations into the craft practices 
of existing professional software developers who work in professional art 
contexts \cite{W10}.

The movement is not purely restriced to the UK. In the US there is a similarly 
led campaign calling to recognise the topic as relevant to all contempary 
sciences. It calls ``Computational Thinking'' a universally applicable 
attitude and skill set everyone, not just computer scientists, would be eager 
to learn and use \cite{Wing06}.

Sonic Pi is one of many projects designed to support both computing and music 
lessons within schools. Sonic Pi is an environment for creating live-coded 
music at a level of complexity which is well suited to a first-programming 
language \cite{sp}. There are many languages in existence which are simply 
enough to also constitute as a good first-programming language, but many do 
not attempt to make themselves an inviting gateway into the realm of technical 
programming. Sonic Pi seeks to provide an exploratory and invigorating 
introduction into programming whilst being complex enough to lead a user 
through into much more complicated programming ideas and use cases with a 
managable learning curve. By presenting a musical system it seeks to break 
down the barrier between the technical elite and the hobbyist.

\subsection{Objectives}
This project's aim is to formalise the concurrency and timing aspects of the 
language and develop a program analysis tool that can be used to identify 
program bugs such as deadlock and thrashing behaviour. The formalism and 
analysis will build on initial work describing timing and
concurrent interaction via effect systems and session types. The project 
combines theoretical aspects of type system and analysis design, as well as 
practical work developing a responsive program analysis engine.

\subsection{Report Structure}
The remainder of this report is broken down as follows:

-Background: We detail the histories of Sonic Pi, Live Programming and
Session Types as fields of research and conclude with work relating to
these subjects.

-Going Forward: The bulk of this section details the expected timeline of
the project and ideas for how to evaluate progress during and at the
conclusion of the project.

\section{Background}
In this section we begin by explaining the ideas and features of Sonic Pi, 
the living coding program that forms the basis of this project. We then 
move on to explain the subject of both Living Programming and Session Types
in further detail and seek to relate them back to the current aims of the
project. We conclude with details on the related work in these areas of
research.

\subsection{Sonic Pi}
% What is Sonic Pi?
Sonic Pi is an imperative live programming language designed as an educational 
first language. It is a ruby-based domain-specific language designed for 
manipulation of synthesisers through time \cite{AB13}. It is currently in its 
second iteration, with the main extension between the two languages being the 
work done to improve the timing system of the project, which is discussed in 
more detail below. Sonic Pi is built on top of the SuperCollider synthesis 
server to enable it to define and manipulate synthesisers in real time; an 
important feature for a musical language. Some of the concepts that Sonic Pi 
is well suited to teach, in direct relevance to the current UK Computing in 
Schools Cirruculum, are conditionals, iteration, variables, functions, 
algorithms and data structures. Sonic Pi also extends beyond these concepts to 
include such things as multi-threading and hot-swapping of code as these are 
likely to be of crucial importance in the future of programming contexts \cite{
AOB14}.

This section first gives a breif description of the Raspberry Pi then explores 
the implementation of Sonic Pi V1.0, mainly to give appropriate context to the 
timing effects system that Sonic Pi V2.0 currently imeplments. There is then 
be some short discussion on the particular features of Sonic Pi V2.0.

\subsubsection{Raspberry Pi}
The Raspberry Pi was developed as a very low cost computer system to enable 
technical experimentation amongst young people who had little other contact 
with computer systems. The idea came about in 2006 as an answer to the 
steadily decreasing levels of pupils applying to take up Computer Science 
after their A-Levels. The reasons for this were attributed to many 
contributing events such as the end of the dot come boom, the focus of IT 
lessons on Microsoft software and building very basic HTML websites and the 
increased availability of out-of-the-box games consoles over the Amigos, BBC 
Micro, Spectrum ZX and Commodore 64 machines that promoted individual 
experimentation so freely in the past \cite{rp}.

The Pi is provided as a bare circuit board costing roughly \$25, able to boot 
into a Linux environment with very little other commerical equipment required. 
The Raspberry Pi Foundation is a non-profit organisation and has sold over a 
million products since 2012. The main objective is to develop genuine 
technical competency by allowing the freedom to experiment with the whole 
system rather than taking the locked box approach of other systems. Learning 
becomes self directed for pleasure rather than at the behest of a mark molded 
system. It is this style of engagement that brought the Raspberry Pi to the 
attention of educational campaigners and has since enabled it to be used so 
successfully within the new movement towards better Computing education within 
Schools. 

\subsubsection{Sonic Pi V1.0}
The initial development time given to Sonic Pi v1.0 was roughly three weeks. 
It was designed largely as a port from the language Overture to focus on 
driving specific educational objectives with the idea in mind to teach a 
target audience of 12-year-olds that had no previous experience with 
programming; it would bring them from the introduction of a computer through 
to the ability to write a full length program over the course of a few weeks. 
Sonic Pi was built as a ruby-based language both through the author's existing 
experience with the language and also to keep it in line with languages 
already used within the industry. Python is well regarded as an educational 
language and given the semantic similarity between Python and Ruby it was easy 
to defend Ruby as a choice of language implementation \cite{AB13}. 



% How is it currently being used?
% What are the features of Sonic Pi?
%% Details of how the relevant features work
%% How Sleep Works {Old and New?}

%% Loops, cue and sync
% Where does this project look to go in the future

% Some interesting remarks about Rasberry Pi itself?


In this report Sonic Pi will be presented in terms of the formalisation of its timing effects and the existing concurrency primitives of the language.


\subsection{Live Programming}
For much of the prevailing history of programming there has been an idea that 
the programmer is inherently separated from the system that they are 
producing. The task of the programmer is to create a system based on some 
formal specification that will take effect at some unknown point in the 
future, and the time between implementation and action has no effect on the 
results that the system will produce. In this way, there is a strong sense of 
separation between the program, process and task domains where the progam is 
the code implementation and specifications, the process is the running of the 
code on a specific machine and the task is the visible real world results****. 
This is a viewpoint that many would not think to challenge as it is natural to 
assume that the methodology of a computer programmer would naturally lend 
itself to implementation of actions that were set for execution in the future 
and, in general, would process a deterministic set of results that can be 
repeatedly used as the users required.

Live programming (also referred to as With Time Programming or Just In Time 
Programming) seeks to apply the improvisional nature of time to the existing 
methodology of the programmer. With this idea it becomes possible to define a 
tigher system of feedback between the program and task domains through means 
of whatever process domain is most suitable. The improvisional nature of the 
acitity also removes the inherent requirement of a specific program 
specification and allows for new level of freedom and creativity in the 
programs being created. 

% What is Live Programming/Just In Time Programming?
Given the nature of Live Programming the languages that invoke it are often 
dynamic language which allow for the flexibility, conciseness and ease of 
development*(5)* to enable the act of live programming to feel as natural as 
standard programming practice.

Live Programming lends itself also to acts of performance, where Live Coders 
will often perform to live audiences, often producing such things as live 
improvisational music or artwork, whilst having some means by which the 
audience will also see the code written at the same time. From a social and 
cultural viewpoint, Live Programming lends itself to breaking down the 
barriers that have been built up between software technology and the creative 
users*(6)*.

% How long has this been an area of research?
% Critics of this as a way of programming?

%% Frame it in the sense that Sonic Pi uses it
%% Education - what makes this so lively

****Programming with Time {Improcess}
*(5)*Living it up with a Live Programming Language
*(6)*The Textual X

\subsection{Session Types}

% What are Session Types
% Concurrency as a field, what does it address?
% Comment on different between binary session types and multi-session types
%% Breif aside that this project currently only expects to use binary
% How do Session Types help with discovering concurrency bugs?

% Outline it in terms of the project
% Concurrency is a hard thing to begin teaching

\subsection{Related Work}
% LP Section, many different languages to detail
% Session Type section? Things that use session types already {Scribble}

%% Finding a crossover would be great
%% If none currently seem to exist, comment on that somewhere
%% Either here or in Final Thoughts

\section{Going Forward}
This section illustrates the plan of action laid out for the rest of the 
project over the next few months. The first section illustrates the rough
idea of how long each implementation of the program features will take, 
with documentation and evaluation time factored in. After that is the 
outline of how the project will be evaluated.

\subsection{Project Outline}

\subsection{Evaluation Outline}

\section{Final Thoughts}

\begin{thebibliography}{9}

\bibitem{number}
  Auther,
  \emph{\LaTeX: Title},
  Publisher, Place,
  Edition,
  Year

\bibitem{naec}
  \emph{http://www.naec.org.uk/events}

\bibitem{sp}
  \emph{http://Sonic Pi.net/}

\bibitem{rp}
  \emph{http://www.raspberrypi.org/about/}

\bibitem{AB13}
  Aaron, S., Blackwell, A.F.,
  \emph{From Sonic Pi to Overtone: Creative Musical Experiences with Domain-Specific and Functional Languages},
  The First ACM SIGPLAN Workshop on Functional Art, Music, Modeling \& Design,
  Boston, Massachusetts, USA,
  ACM, pp. 35-46,
  2013

\bibitem{AOB14}
  Aaron, S., Orchard, D., Blackwell, A.F.,
  \emph{Temporal Semanics for a Living Coding Language},
  Proceedings of the 2nd ACM SIGPLAN International Workshop on Functional Art, Music, Modeling \& Design,
  Boston, Massachusetts, USA,
  ACM, pp. 37-47,
  2014

\bibitem{BAD14}
  Blackwell, A.F., Aaron, S. and Drury, R., 
  \emph{Exploring Creative Learning for the Internet of Things Era},
  In B. du Boulay and J. Good (Eds) Proceedings of the Psychology of Programming Interest Group Annual Conference, 
  pp. 147-158,
  (PPIG 2014)

\bibitem{BC05}
  Blackwell, A.F. and Collins, N.,
  \emph{The Programming Language as a Musical Instrument},
  In Proceedings of the Psychology of Programming Interest Group Annual Conference,
  pp. 120-130,
  (PPIG 2005)

\bibitem{Ch12}
  Church, L., Rothwell, N., Downie, M., deLahunta, S. and Blackwell, A.F.,
  \emph{Sketching by Programming in the Choreohraphic Language Agent},
  In Proceedings of the Psychology of Programming Interest Group Annual Conference,
  pp. 163-174,
  (PPIG 2012)

\bibitem{DfEO13}
  Department for Education and Ofsted,
  \emph{ICT in schools: 2008 to 2011},
  Piccadilly Gate,
  Manchester,
  110134,
  2013

\bibitem{DfE13}
  Department for Education,
  \emph{National curriculum in England: computing programmes of study (key stages 1 - 4)},
  2013

\bibitem{W10}
  Woolford, K., Blackwell, A.F., Norman, S.J. \& Chevalier, C.,
  \emph{Crafting a Critical Technical Practice},
  Leonardo 43(2),
  202-203,
  2010

\bibitem{Wing06}
  Wing, J.M.,
  \emph{Computational Thinking},
  Communication of the ACM,
  Vol. 49, pp. 33-35,
  2006

\end{thebibliography}

\end{document}