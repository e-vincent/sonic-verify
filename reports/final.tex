\documentclass[11pt, abstracton, twoside, titlepage=true]{scrartcl}
\usepackage{a4, fullpage}
\usepackage[usenames,dvipsnames]{color, xcolor}
\usepackage{listings, caption, graphicx}
\usepackage{multicol, titlesec}
\usepackage{fancyvrb}
\usepackage{enumitem}
\usepackage{amsmath, amssymb}
\usepackage{afterpage}
\usepackage{lipsum}
\usepackage[automark]{scrpage2}

\pagestyle{scrheadings}
\clearscrheadfoot
\setheadsepline{0.5pt}

\ihead{\headmark}{}
\ofoot[\pagemark]{\pagemark}

\titleformat
	{\paragraph}
	{\normalfont\normalsize\bfseries}
	{\theparagraph}{0.5em}{}
\titlespacing*
	{\paragraph}
	{0pt}
	{1ex plus 0.5ex minus .2ex}
	{1ex plus .2ex}

\renewcommand{\lstlistingname}{Snippet}
\renewcommand{\titlehead}{\textbf}

\DeclareGraphicsExtensions{.pdf,.png,.jpg}
\DeclareCaptionFont{white}{\color{white}}
\DeclareCaptionFormat{myformat}{#1#2#3\hrulefill}

\captionsetup{format=myformat}
\captionsetup[lstlisting]{
	position = bottom,
	format = myformat
}

\lstset{
	basicstyle   = \ttfamily\color{black}	
}

\lstdefinestyle{sonicpi}{
	language = ruby,
	basicstyle   = \ttfamily\color{black},
	keywordstyle = \ttfamily\color{blue},
	morekeywords = {
		define, load_samples, each
	},
	morekeywords = {[2]{
		play, sleep, cue, sync,
		release, sample, choose, use_synth, cutoff, rrand, chord,
		ring, scale, puts, use_synth_defaults, amp, attack,
		sustain, note, octave, mod_phase, mod_invert_wave, mod_range,
		rate
	}},
	morekeywords = {[3]{
		in_thread, live_loop, loop, times
	}},
	keywordstyle = {[2]{\color{magenta}}},
	keywordstyle = {[3]{\color{ForestGreen}}},
	commentstyle = \ttfamily\color{Gray},
	stringstyle  = \color{orange}
}

\newenvironment{myitemize}{ 
\begin{itemize}
    \setlength{\itemsep}{1pt}
    \setlength{\parskip}{0pt}
    \setlength{\parsep}{0pt}     
}{ 
\end{itemize}                  
}

\newenvironment{blockquote}{
	\par
	\medskip
	\leftskip=4em\rightskip=3em
	\noindent\ignorespaces
}{
	\par\medskip
}

\def\therefore{
	\leavevmode
	\lower0.4ex\hbox{$\cdot$}
	\kern-.5em\raise0.7ex\hbox{$\cdot$}
	\kern-0.55em\lower0.4ex\hbox{$\cdot$}
	\thinspace
}

\lstset{
  mathescape,         
  literate={->}{$\rightarrow$}{2}
           {ε}{$\varepsilon$}{1}
}

\setlength{\headsep}{0.5in}
\setlength{\parskip}{0.4cm}
\setlength{\parindent}{0cm}

\let\endtitlepage\relax

\begin{document}

\subject{{\large Department of Computing \\ Imperial College London} \\
\LARGE{Individual MEng Project}}
\title{Musical Concurrent Programming \\ With Sonic Pi}
\author{{\LARGE Eleanor Vincent} \\ {\large Supervisor: Nobuko Yoshida}}
\date{\today} 

\maketitle{\thispagestyle{empty}}

\begin{center} 
	\includegraphics[width=0.50\textwidth]{images/crest.jpg}
\end{center}

\afterpage{\thispagestyle{empty}\null\newpage}
\newpage

\thispagestyle{empty}
\begin{abstract}
The report details the improvements made to Sonic Pi: a musical live coding 
language. With this style of live, interactive improvisation, time becomes a crucial 
factor in giving an enjoyable, effective performance. Sonic Pi combines this with 
multiple concurrency primitives that affect the interactions between the different 
running threads. The current problem with Sonic Pi is the lack of ability to verify 
the program before running it; during a live performance this can be disastrous. The 
challenges this present involve the difficulty in handling concurrent code and how 
to define the concept of time in the context of musical coding. When musical pieces 
become very long and complex the length of time the piece runs for, and which music 
threads interact with which, become very difficult to reason about.

In this report we develop a lightweight static analysis and typing tool for Sonic Pi 
program. We develop a formal specification of the temporal behaviour of Sonic 
Pi, and describe our implementation of this system. Alongside this 
we apply the field of session types to the underlying communication structure of 
Sonic Pi, formalise its ideas, and present the implementation of this typing 
system. By running the systematic testing suite we have designed, we prove the 
correctness of our analysis of the program against these two formalisations and 
also comment on the overall performance of the additions. We evaluate the results 
of our analysis in the context of an educational language. Many of the features 
implemented lend themselves to personal exploration while others are well suited 
to the goal of educating young people in the realms of concurrency and temporal 
reasoning.
\end{abstract}
\afterpage{\thispagestyle{empty}\null\newpage}
\newpage

\renewcommand{\abstractname}{Acknowledgements}

\thispagestyle{empty}
\begin{abstract}
Firstly, I wish to thank my supervisor, Nobuko Yoshida, both for proposing this 
project, and for being a constant source of motivation, inspiration and support; even 
through her recovery earlier this year. Secondly, I extend a warm thanks to Dominic 
Orchard, whose help and feedback throughout this whole project has been invaluable. 

I would like to give a special mention to Sam Aaron as thanks for building and 
open-sourcing Sonic Pi as it has been an absolute joy to work with. My thanks also 
go out to my family, for their unwavering support of my degree, my housemates, for 
keeping me healthy, and my other friends, for their company.
\end{abstract}
\afterpage{\thispagestyle{empty}\null\newpage}
\newpage

\thispagestyle{empty}
\tableofcontents
\newpage

\section{Introduction}
\thispagestyle{empty}

\subsection{Motivation}
% need more on the concurrency verification motivation!!
% easy to feed into how difficult that can be to teach?
For many years it has not been a requirement that children should learn much
about the vast field of computing during their formative years in UK 
Education. In 2012, the Government began to re-recognise the signifiance of 
Computer Science, after the last Cirriculum Project had been shut down in 1991, 
and has replaced the National ICT curriculum with a revised 
Computing cirruculum. The new Computing Program of Study \cite{DfE13} aims to 
enable pupils to understand the world of computing, giving them the ability to 
think logically and apply the fundamental principles of the discpline to their 
real-world environments.

The movement is not purely restricted to the UK. In the US there is a similar 
campaign calling to recognise the topic as relevant to all contempary 
sciences. It calls ``Computational Thinking'' a universally applicable 
attitude and skill set everyone, not just computer scientists, would be eager 
to learn and use \cite{Wing06}. A recent Microsoft article\footnote{Part of the 
\#WeSpeakCode campaign.} claims that 75\% of students in the Asia Pacific region
wish that programming could be offered as a core subject in their schools \cite{micro}.

There is a general struggle within the humanities courses available in UK schools 
to remain relevant in light of a quickly developing technical world. Music schemes 
within the UK frequently report having suffered funding cuts, and education is 
focused on learning a variety of specific instruments, with little focus on 
musical technology.

There is an increasing recognition of the power of programming amongst other 
disciplines. A growing hacker and maker movement has been making programming a 
much more accessible skill \cite{BAD14}; it presents itself as a viable and 
useful hobby amongst a vast range of ages and professions, largely due to the 
increasing availability of resources. Existing research has examined the 
viability of programming tools for professional artists, as reported at PPIG 
\cite{Ch12,BC05}, and the craft practices of existing professional software
developers who work in professional art contexts \cite{W10}. A powerful example 
of this is the popular uptake of Minecraft in both educational and personal areas. 
Many uses focus on playing Minecraft on maps designed to teach various scientific 
concepts \cite{mine1,mine2} but there is the ability to learn programming through 
mod development itself. Official statistics are difficult to locate but a popular 
game download website listed 2,000+ Minecraft mods, and 14,000+ texture packs for 
download; the most popular mod has almost 3 million downloads \cite{crush}, proving 
its popular reach. 

In parallel with this we have entered an age where concurrency and 
distribution have become some of the most pressing issues in computing. As we 
begin to process hugely increased amounts of data the ideas of parallelism and 
exploiting concurrency are becoming more and more pronounced. In this day and age
it is almost impossible to find a person who does not own a mobile, most likely a 
`smart' device, which relies on communication with other such devices to function 
effectively. 

Concurrency has been an active field of research since the 1960s, producing
many formalisms over that time which have improved the field of computing. One 
such formalism is that of process calculi, a field that $\pi$-calculus is a strong
member of. Another type of process calculi, CSP, was highly influential in 
bringing to light such languages as Go: a relatively young language but praised
for its concurrency primitives and performance. Go can be found in the backend
of many notable technology companies. Models of concurrency are being used for
reasoning and specification as well as at all stages of the development life
cycle: design, implementation, testing, etc...

Whilst there has been much in the field to support a programmer as they produce
type-safe code, the same cannot be said for communication-safe code. Concurrency
bugs such as deadlocks and thrashing behaviour can be as hard to locate now as
ever before. Some tools and techniques exist to handle these situations, such as
mutexes and locks \cite{T95}, but even then there is no guarantee
that it will work. 

\subsection{Objectives}
In the current version of Sonic Pi \cite{sp} there is no means to verify the 
temporal behaviour or the communication patterns of the program. This is to say, 
users must judge themselves how long the current program is, that the program is 
timed correctly, and that the program is free from concurrency bugs. This can be 
difficult enough in a static environment; in Sonic Pi there is the additional 
requirement to do all of these things during a live performance (studio audience 
included)! The first point can be difficult for novice programmers, potentially 
moreso for those with little musical experience. The latter can be difficult for 
even some experienced programmers due to the non-deterministic nature of some 
concurrency bugs.

We aim to develop a lightweight library to analysis the temporal behaviour of a 
Sonic Pi program. There are several interesting challenges in terms of 
the heuristics used when parsing the program and what solutions will lead to the 
best possible results. We plan to utilise the speed of the platform and program 
architecture and efficient design; it is important that the project is still 
able to run on Sonic Pi's target architecture. Temporal behaviour should be able 
to handle basic sequencing, funtion calls, nesting and variable time operations. 

We aim to build an overview of the communication patterns within a Sonic Pi program 
by utilising the theor of Session Types. Session Types will
enable us to define a clearly understandable structure to the communication
protocol of the program; this will enable the tool to identify difficult bugs
such as deadlocks and thrashing behaviour. In identify problems in this manner,
we will be able to reduce the number of crashes developers experience whilst 
writing code `on-the-fly'. 

The final requirement will be for the analysis to run within the Sonic Pi IDE, 
and to run quickly enough to be utilised during a real perforamnce environment. 
The challenge here is to make the analysis sufficiently lightweight enough not to
slow the IDE while the synth servers are working. 

\subsection{Contributions}
We have produced a lightweight shared library, often affectionately referred to 
as Sonic-Verify, that enables a variety of different analysis of a program 
written for the Sonic Pi IDE. The project analyss the timing effects of the 
program, outputting such information as the total `virtual time' elapsed by the 
program and more specific detail, such as function durations. This implementation
of a formal timing system presents an improved environment in which developers
may code more asthetically pleasing music in a live environment.

This is the first instance of the theory of Session Types being applied in 
a (musical) live programming context. With this analysis we are able to output 
the decomposed types of all processes in the program and the associated global 
type, presenting an intuitive description of the interactions of processes 
within the source code. This presents an exciting new realm of possibility in
terms of concurrency verification and the application of Session Types in 
existing distributed paradigms. We also present two new extentions to Multi-Party 
Session Types based on the conceps of replication types and broadcasting, as they 
appear in Sonic Pi. We also present a method of \emph{Global Inference}, an area 
of Multi-Party Session Types that is still largely undeveloped.

We have implemented a set of tests that systematically verifies the results that
our project produces; it covers a range of features starting from the most basic
of sequential programs through into full music sources, with complex function
call structures and multiple interacting process threads. Finally, we have fitted
this library into the existing Sonic Pi IDE in a manner that is beneficial to both
new coders and `old hats' alike. 

\subsection{Report Structure}
The remainder of this report is broken down as follows:

\begin{itemize}
	\item Background: We detail the basic concepts of Sonic Pi and its timing 
	effects, and Session Types to give an adequate foundation for the work 
	presented in this project. We also explore the field of Live Programming 
	as part of the background research.
	\item Related Work: We detail the existing work in the field Live Programming
	with a particular focus on those with musical contexts. We also detail some
	of the existing applications of Session Types, outlining the approaches and
	differences therein.
	\item Design \& Technical Implementation: Here the chapter is broken into
	two core sections, focusing on both the timing effects of Sonic Pi and
	the session types of Sonic Pi in turn. We discuss chosen languages and key
	libraries used in the design of the program and then move into general
	implementation and interesting algorithms and code from each section. The
	chapter concludes with a brief discussion of the architecure of the Sonic 
	Pi IDE and how we chose to integrate our verification library.
	\item Evaluation: Here we evaluate the success of the project. We discuss
	the limitations of the current state of the project and outline qualitive
	and quantative metrics for the project, including a systematic test of
	the correctness of our approach.
	\item Conclusion: Here we present possible improvement for the project in
	the future work section before closing with a summary of those objectives
	achieved.
\end{itemize}
\newpage

\section{Background}
\thispagestyle{empty}
In this section we begin with a short history of Computing in Schools followed 
by explaining the ideas and features of Sonic Pi, the live coding IDE that forms 
the basis of this project. We then move on to explain the subject of both Live 
Programming and Session Types in further detail and seek to relate them back to 
the current aims of the project.

\subsection{Live Programming}
For much of the prevailing history of programming there has been an idea that 
the programmer is inherently separated from the system that they are 
producing. The task of the programmer is to create a system based on some 
formal specification that will take effect at some unknown point in the 
future, and the time between implementation and action has no effect on the 
results that the system will produce. In this way, there is a strong sense of 
separation between the program, process and task domains where the progam is 
the code implementation and specifications, the process is the running of the 
code on a specific machine and the task is the visible real world results
\cite{SG10}. This is a viewpoint that many would not think to challenge as it 
is natural to assume that the methodology of a computer programmer would 
naturally lend itself to implementation of actions that were set for execution 
in the future and, in general, would process a deterministic set of results 
that can be repeatedly used as the users required.

Live programming (also referred to as With Time Programming or Just In Time 
Programming) seeks to apply the improvisional nature of time to the existing 
methodology of the programmer. With this idea it becomes possible to define a 
tigher system of feedback between the program and task domains through means 
of whatever process domain is most suitable. The improvisional nature of the 
acitity also removes the inherent requirement of a specific program 
specification and allows for new level of freedom and creativity in the 
programs being created. 

Given the nature of Live Programming the languages that invoke it are often 
dynamic languages which allow for the flexibility, conciseness and ease of 
development \cite{McD07} to enable the act of live programming to feel as 
natural as standard programming practice.

Live Programming lends itself also to acts of performance, where Live Coders 
perform to live audiences, often producing such things as live 
improvisational music or artwork, whilst having some means by which the 
audience will also see the code written at the same time. For a Live Coder 
there is frequently no desire to create a final software product or even a set 
musical score as live programming is about the experimentation rather than the 
manufacture. From a social and cultural viewpoint, Live Programming lends 
itself to breaking down the barriers that have been built up between software 
technology and the creative users \cite{McL13}.

It is interesting that the user's level as a musician will have specific 
impacts on their experience with the systems in use in the context of live 
programming and audio languages. Musical scores provide an implicit time 
representation whilst most musical languages tend to use explicit temporal 
structures. This generally makes the representation of rhythm within the 
language guarantee only the ordering of the notes and not the time elapsed 
between playing each one; the temporal structure provides no clear guarantee as 
to execution length. In terms of musical experience, non-experienced 
users may be able to identify an odd sound within this system but be unable to 
pinpoint exactly what causes the problem. 

The question of time and temporal semantics within computing has been in 
existence for some time, but live programming as a field is a very young 
research area, with most of the popular languages appearing over the last 
decade. This comes as part of a wider movement of programming reaching more of 
the general populace and as it is applied with more and more creative outlets 
in mind, this separate way of thinking about the programming environment is 
likely to produce more interesting projects as time develops. 

\subsection{Sonic Pi}
Sonic Pi is an imperative live programming language designed as an educational 
first language. It is a Domain-Specific Language based on Ruby designed for 
manipulation of synthesisers through time \cite{AB13}. It is interpreted through 
a VM rather than being directly compiled. It is currently in its 
second iteration, with the main extension between the two languages being the 
work done to improve the timing system of the project, which is discussed in 
more detail below. Sonic Pi is built on top of the SuperCollider synthesis 
server to enable it to define and manipulate synthesisers in real time; an 
important feature for a musical language. Some of the concepts that Sonic Pi 
is well suited to teach, in direct relevance to the current UK Computing in 
Schools Cirruculum, are conditionals, iteration, variables, functions, 
algorithms and data structures. Sonic Pi also extends beyond these concepts to 
include such features as multi-threading and hot-swapping of code as these are 
likely to be of crucial importance in the future of programming contexts \cite{
AOB14}. It also presents an inviting environment to teach concepts such as 
parallelism and concurrency; a core strength of the program as these can be
notoriously difficult concepts to explain.

\subsubsection{Computing in Schools}
Some of the earliest significant pieces of work towards the 
education of computing started with the invention of Logo, an adaptation of 
the LISP language, most remembered for its use of ``turtle graphics''. Some 
time after this came the Computers in the Cirriculum Project, funded from 
1972 to 1991 by the Schools Council with addition aid from the Microelectronics 
Education Programme introduced in 1981. The first microcomputers appeared in 
both UK Primary and Secondary Schools in 1979. The Commodore Pets aided both 
spelling and arthimetic practice as well as the ability to teach either BASIC or 
Logo. The 1980s saw a huge amount of legislative reform and technical efforts 
to give young people the ability to work with computers. Unfortunately, over the 
course of the late 80's and early 90's, this focus on programming ability gave 
way to simply the education of practical use of existing computer software; 
there was little to no focus on the programming fundamentals behind 
these applications \cite{naec}.

In 2013, Ofsted published a report documenting their findings relating to ICT 
in UK schools from 2008 to 2011 and found that in half of all secondary 
schools, school leavers had not been given adequete education to move into a 
technical career in their future. In 2007, 81,100 pupils were enrolled in the 
ICT GCSE but this had fallen to 31,800 pupils by 2011 \cite{DfEO13} with a 
notable lack in the education of key skills such as computer programming 
itself. This lack was found to be as much a lack in knowledge from the 
teachers as much as the cirriculums failure to address the issues.

\subsubsection{Raspberry Pi}
The Raspberry Pi was developed as a very low cost computer system to enable 
technical experimentation amongst people who have little access to computers 
they could tinker with or expose the programming layer. The idea came about in 
2006 as an answer to the steadily decreasing levels of pupils applying to take 
up Computer Science after their A-Levels. The reasons for this were attributed 
to many contributing events such as the end of the ``dot com'' boom, the focus 
of IT lessons on Microsoft software and building very basic HTML websites and 
the increased availability of out-of-the-box games consoles over the Amiga, BBC 
Micro, Spectrum ZX and Commodore 64 machines that promoted individual 
experimentation so freely in the past \cite{rp}.

The Pi is provided as a bare circuit board costing roughly \$25\footnote{About 
\pounds16.}, able to boot 
into a Linux environment with very little other commerical equipment required. 
The Raspberry Pi Foundation is a non-profit organisation and has sold over a 
million products since 2012. The main objective is to develop genuine 
technical competency by allowing the freedom to experiment with the whole 
system rather than taking the locked box approach of other systems. Learning 
becomes self directed for pleasure rather than at the behest of a mark molded 
system. It is this style of engagement that brought the Raspberry Pi to the 
attention of educational campaigners and has since enabled it to be used so 
successfully within the new movement towards better Computing education within 
Schools. 

\subsubsection{Sonic Pi V1.0}

\begin{multicols}{2}
Sonic Pi V1.0 was designed to port the language Overtone to the Raspberry Pi in 
order to focus on driving specific educational objectives; the idea in mind was 
to teach a target audience of 12-year-olds that had no previous experience with 
programming; it would bring them from the state of ``this is a computer'' through 
to the ability to write a full length program, that generated satisfying music, 
over the course of a few weeks. Sonic Pi is a Ruby-based language both due to 
the author's existing experience with the language and to keep it in line with 
the languages already used within the industry. Python is well regarded as an 
education language and the semantic similarity between the two makes Ruby easy 
to defend as a language choice.
\\

\begin{minipage}{0.5\textwidth}
	\begin{minipage}[t]{\textwidth}
		\begin{lstlisting}[style = sonicpi]
	play 60
	play 61
	play 62
		\end{lstlisting}
		\captionof{lstlisting}{Chord Form: Successive Notes}
	\end{minipage}

	\begin{minipage}[t]{\textwidth}
		\begin{lstlisting}[style = sonicpi]
	play 60
	sleep 0.5
	play 61
	sleep 0.5
	play 62
		\end{lstlisting}
		\captionof{lstlisting}{Arpeggio Form: Sleep Separation}
	\end{minipage}
\end{minipage}

\end{multicols}

The immediate feature that Sonic Pi focuses on is sequential ordering in 
imperative programs; demonstrated in musical theory by the sequential playing 
of notes. Code Snippets 1 and 2 demonstrate two small Sonic Pi programs. The 
first demonstrates the situation where the MIDI notes would be played together. 
Sonic Pi v1.0 takes advantage of fast clockspeeds of modern processors in order 
to assume the sequence of instructions would execute quickly enough 
to sounds together. In order to separate them into sequential notes they must 
be separated with a sleep instruction as demonstrated by Snippet 2. The notation 
for sleep in Sonic Pi V1.0 is similar to that of the POSIX sleep command 
\cite{IG13}. The default sound that Sonic Pi uses in these contexts is a 
pleasant bell sound from the SuperCollider synthesiers.

It can be seen that the ability to skip the arguably verbose nature of 
structured syntax makes it much more relevant in the contexts of a classroom; 
pupils are able to start creating meaningful programs with much more speed. From 
the point of view of providing any future debugging interfaces, it has a small 
benefit in removing the code overhead for finding and correcting such small 
syntactical issue such as missing \texttt{;}. 

These semantics work well in an educational context but do not allow for the 
correct timing of musical notation which puts them at odds with user 
expectations. To demonstrate this more concretely, below are two further code 
snippets demonstrating Sonic Pi's basic threading abilities. Sampling is a 
feature of Sonic Pi V2.0 but is presented here in the context of V1.0 
programs as the two versions are syntactically similar.

\begin{multicols}{2}
In Snippet 3 the desired outcome is to play MIDI note 60 together with the 
drum kick with an interval of 0.5s between each hit. Unfortunately, this does 
not take into account the execution time of each statement; there is a short 
amount of execution time for each line of code, plus the desired sleep of 0.5s.
Because of this the rhythm will gradually shift with each loop as the 
clock time and the ``virtual time'' becomes further out of sync. The 
actually length of time each line execution takes is variable between 
processors depending on their speed and overall load. Regardless, we can see 
that each loop in Snippet 3 will actually take longer than the desired 0.5s.

This is further apparent when running Snippet 4. The desired outcome is to 
play the drum kick every second and the MIDI 60 note at every half second, so 
the two notes will play at the same time every second MIDI note, the threads 
remain synchronised. On running the threads it is quickly apparent that this 
is not the true result; due to differing execution times the thread rhythms 
drift apart very quickly.

	\begin{minipage}{0.5\textwidth}

		\begin{minipage}{\textwidth}
			\begin{lstlisting}[style = sonicpi]
loop do
    play 60
    sample :drum_heavy_kick
    sleep 0.5
end
			\end{lstlisting}
			\captionof{lstlisting}{Repeating Bass and Drum}
		\end{minipage}

		\begin{minipage}{\textwidth}
			\begin{lstlisting}[style = sonicpi]
in_thread
    loop do
        play 60
        sleep 0.5
    end
end

in_thread
    loop do
        sample :drum_heavy_kick
        sleep 1
    end
end
			\end{lstlisting}
			\captionof{lstlisting}{Concurrent Threads}
		\end{minipage}

	\end{minipage}

\end{multicols}

The $play$ and $sample$ calls are asynchronous and this compounds the present 
timing issue due to additional costs in sending and interpreting the messages. 
These issues are summarised in Figure 1. The left-most column represents the 
real computation time of the statement whilst the right-most column shows the 
point at which each statement would be run. Each statement duration is unique 
as processor speed and system load variations affect the duration of each 
statement separately. The durations are therefore non-deterministic in nature 
and also not consistent across different runs of the same program.

\begin{figure}[ht]
	\centering
	\includegraphics[width=0.9\textwidth]{images/sonic-one.png}
	\caption{Timing in Sonic Pi V1.0 \cite{AOB14}}
\end{figure}


\subsubsection{Sonic Pi V2.0}

\paragraph{Timing Sleep}
Sonic Pi V2.0 sought to address this temporal issue and introduces the 
interesting concept of a \emph{time system} and associated \emph{time safety}, 
which draw an analogy from the familiar programming concepts of \emph{type 
systems} and \emph{type safety}. V2.0 maintains syntactic 
compatability with V1.0, so it as conceptually useful to apply in the 
educational context as before. The power behind V2.0 is that its temporal 
semantics now react as a user would expect them to, making it a viable program 
for the musically experienced who have some concept of what they want to 
achieve as well as for those beginners who desire to learn.

In Sonic Pi V2.0 the sleep command no longer mimics the POSIX command as 
commented earlier. Instead the programming model allows a separation of the 
ordering of effects from the timing of effects. Snippet 5 shows a simple 
example that combines the different kind of effects; parallel, timed and 
ordered effects. It also demonstrates further syntactic abilities of V2.0 as 
we are now treating the code snippets as V2.0 programs.

\begin{minipage}{\textwidth}
	\begin{lstlisting}[style = sonicpi]
		  play :C ; play :E ; play :G
		  sleep 1
		  play :F ; play :A ; play :C
		  sleep 0.5
		  play :G ; play :B ; play :D
	\end{lstlisting}
	\captionof{lstlisting}{V2.0 Program: Playing Three Chords} \label{chords}
\end{minipage}

Each chord line demonstrates Sonic Pi's ability to play notes in parallel, the 
system still taking advantage of the capabilities of modern processors\footnote{
Snippet \ref{chords} demonstrates the ordered effect, as the three chords are 
played in the order written, and also Sonic Pi's abiltity to run with or without 
semi-colons.}. \texttt{sleep} now acts as a ``temporal barrier'' between 
statements; it works by blocking computation from proceeding until the given time 
has elapsed \emph{since the program began running}. It does \emph{not} block from 
the end of the notes played. In terms of Snippet 5, this means that the second 
chord is played once one second has elapsed and the third chord will not be played 
until 1.5 seconds have elapsed. One can think of \texttt{sleep} as an ``at 
least'' timing. In other words, once \texttt{sleep t} has been evaluated, we 
can state that at least \texttt{t} seconds have elapsed since the last \texttt{
sleep} statement was called. 

\begin{figure}[ht]
	\centering
	\includegraphics[width=0.8\textwidth]{images/sonic-two.png}
	\caption{Timing in Sonic Pi V2.0 \cite{AOB14}}
\end{figure}

The semantics introduced here are achieved by implementing the concept of 
``virtual time''. In Sonic Pi V2.0, virtual time is a 
thread-local variable that is only advanced by a new \texttt{sleep} command, 
which means that the programmer has explicit control over the timing of the 
program. Each thread maintains access to both the real time elapsed and the 
virtual time elapsed whilst running a given program and used the virtual time 
variable to scheduled requested effects. In order to keep time with the 
explicit timing requirements of the program the \texttt{sleep} command will 
take account of the execution time between the last \texttt{sleep} statement 
and the current execution point. Referring back to Snippet 5, at the point at 
which the program executes the second \texttt{sleep} command, if execution 
of the F major chord took 0.1s of execution time then the program will only 
sleep for 0.4s. This is to ensure there is no rhythm drift; the only overhead 
in the rhythm is from the play statements following the last \texttt{sleep} 
executed. This is demonstrated visually with Figure 2.

To deal with the non-deterministic execution times within a sleep barrier, and 
also to deal with the time cost for the synthesiser to schedule output 
effects, a constant \texttt{scheduleAheadTime} value is added to the current 
virtual time for all asynchronously scheduled effects. If the 
execution time between \texttt{sleep} commands never exceeds this value then 
the temporal requirements of Sonic Pi are met. 

It is possible that the time between \texttt{sleep} commands may over-run - 
this may be common in the event someone requested a short \texttt{sleep} time 
such as 0.1s or even 0.05s. In this case, the described programming model is 
not useful for providing hard deadlines but can function with ``soft'' 
deadlines (along the vein of Hansson and Jonsson \cite{HJ94}). In the event a 
thread falls behind in execution time then the user is given explicit 
warnings. Should the amount of time the thread is behind exceed a specific fall
-behind value, as defined within Sonic Pi, then the system will stop that 
thread and throw a time exception. This provides essential temporal 
information about the program and its behaviour to the users. This is a 
positive in terms of educational use but is something a live coder will aim to 
avoid during a real performance. This feature also provides a safety mechanism 
against common errors such as placing isolated \texttt{play} calls between 
\texttt{sleep} commands which would have the problem of taking up all of the 
system resources; instead the thread self-terminates and allows any other 
threads to continue executing.

\paragraph{Threading}

Threading has already been introduced briefly; this section focuses on the 
further improvement that Sonic Pi V2.0 brings to its threading primitive. 
Within V2.0 there are multiple commands which allow for easy synchronisation 
of threads whilst the program is running. Threads also implement thread-
based inheritance wherein they take all of the synthesiser settings of the 
thread they were spawned from.

The keyword \texttt{in\_thread} enables Sonic Pi users to run pieces of 
code concurrently. Threads can be named as shown in Snippet 6 in a similar way 
to how functions are named in Sonic Pi. The simple nature of its loop syntax 
enables this to be as easy a concept to grasp as the previously defined 
temporal semantics. These threads alone do not allow for the live coding that 
Sonic Pi has been created for; a developer must combine each thread with a loop
in order to produce constant sound. Sonic Pi V2.0 provides the \texttt{live\_loop}
keyword to capture the same effect with less writing.

An interesting point to note about Sonic Pi V2.0 is that actually running the 
program starts the current program in another thread. Because of this one can 
actually press run multiple times and have the program layered over the top of 
itself. There are no particular synchronisation primitives defined over these 
``global'' threads, but it makes for an interesting performance if timed 
correctly.

\begin{minipage}{\textwidth}
	\begin{lstlisting}[style = sonicpi]
    in_thread(name: :bass) do
        loop do
          use_synth :prophet
          play chord(:e2, :m7).choose, release: 0.6
          sleep 0.5
        end
    end

    in_thread(name: :drums) do
        loop do
          sample :elec_snare
          sleep 1
        end
    end
	\end{lstlisting}
	\captionof{lstlisting}{Named Threads}
\end{minipage}

\begin{minipage}{\textwidth}
	\begin{lstlisting}[style = sonicpi]
    live_loop :foo do
        play :c1, release: 8, cutoff: rrand(70, 130)
        sleep 8
    end
	\end{lstlisting}
	\captionof{lstlisting}{Live Loops}
\end{minipage}

While running a program, Sonic Pi's \texttt{live\_loops} will automatically update 
the program without skipping any beats. This gives the users a great amount of 
freedom to experiment with different sounds without having to reset the program
with every small edit, as is the case with standard 
programming. This feature, however, is directly affected by the previously 
described temporal semantics; loops can easily become out of time with 
each other during a performance. To combat this, Sonic Pi provides 
synchronisation semantics in the form of the \texttt{cue} and \texttt{sync} 
commands. Each time a \texttt{live\_loop} loops it will generate a new \texttt{cue} 
event which we are able to \texttt{sync} on to. \texttt{cue} is an asynchronous,
non-blocking operation and \texttt{sync} is a blocking operation.

\begin{multicols}{2}
The Snippets to the right demonstrate a rough workflow of how a user would use 
the \texttt{cue} and \texttt{sync} features. To begin with we assume the loops 
\texttt{:foo} and \texttt{:bar} are out of time.

We can start to fix the situation by 
changing the sleep time in \texttt{:foo} to 0.5s. Most likely, this will still 
sound incorrect. This is because the two loops are likely to now be out of 
time with each other. Both \texttt{:foo} and \texttt{:bar} are producing \texttt{
cue} events, but these are not being used and so both loops are running with 
no regard to the other. We can fix this by \emph{syncing} one thread to the 
other, so that it will only fire when the other thread has looped (since
a \texttt{live\_loop} will send a \texttt{cue} message at the start of each loop).
In this case we have synced \texttt{:bar} onto \texttt{:foo}'s \texttt{cue} 
message.

This gives way to some very obvious deadlock scenarios (example shown in Snippet 10) 
that users must avoid. Here we have created two \texttt{live\_loop}s, \texttt{:foo}
and \texttt{:bar}. Both threads finish with a blocking call, waiting for a
\texttt{cue} message from the other loop. The issue here is, since both loops 
are blocked at the same time, neither loop will ever trigger another \texttt{cue}. 

	\begin{minipage}{0.5\textwidth}

		\begin{minipage}{\textwidth}
			\begin{lstlisting}[style = sonicpi]
live_loop :foo do
    play :e4, release: 0.5
    sleep 0.4
end

live_loop :bar do
    sample :bd_haus
    sleep 1
end
			\end{lstlisting}
			\captionof{lstlisting}{Out of Sync Live Loops} \label{outofsync}
		\end{minipage}
		\begin{minipage}{\textwidth}
			\begin{lstlisting}[style = sonicpi]
live_loop :foo do
    play :e4, release: 0.5
    sleep 0.5
end

live_loop :bar do
    sync :foo
    sample :bd_haus
    sleep 1
end
			\end{lstlisting}
			\captionof{lstlisting}{Synced Live Loops} \label{synced}
		\end{minipage}

	\end{minipage}
\end{multicols}

\begin{minipage}{\textwidth}
	\begin{lstlisting}[style = sonicpi]
    live_loop :foo do                  live_loop :bar do
        play :e4, release: 0.5             sample :bd_haus
        sleep 0.5                          sleep 1
        sync :bar                          sync :foo
    end                                end
	\end{lstlisting}
	\captionof{lstlisting}{Deadlock - both loops waiting for cue}
\end{minipage}

\paragraph{The IDE}
The IDE is as much a part of the educational experience as the language 
itself. Sonic Pi features a bespoke environment that contains only the bare 
minimum features required to enable pupils to start coding quickly and with 
minmal confusion. For this reason the IDE only consists of five key components:
\newpage
\begin{itemize}
	\item Control Buttons (Musical/Interface)
	\item Workspace Tabs
	\item Editor Pane
	\item Information Pane
	\item Error Pane
\end{itemize}

\begin{figure}[t]
	\centering
	\includegraphics[width=\textwidth]{images/sonic-ide.png}
	\caption{Sonic Pi V2.1.1 IDE - Apple Mac View} \label{macview}
\end{figure}

Sonic Pi originally choose not to have a file save system as the workspaces 
automatically save the current session. There were no other usual IDE features 
such as project structure, macro system, refactoring wizard, etc... This is to 
keep the system as simple as possible to make it quick for young pupils and non
-IT teachers to get to grasps with.

Since then, in Sonic Pi V2.1.1, it has added the ability to save the current 
workspace to your file system, the ability to record what is playing and save 
as a .wav file type, basic formatting buttons, the ability to import other 
music files as custom samples, information and tutorial sections, all whilst 
remain simplistic and easily useable. Features are non-intrusive but 
incredibly useful in terms of allowing those pupils who thrive with the system 
to continue to experiment outside the scope of the regular cirruculum lessons.

\subsection{Session Types}
Over recent years there has been a massive increase in the amount of 
communication based systems, from network protocols over the Internet to server
-client systems in local area networks to distributed applications on a global 
scale. The crucial observation amongst all of these systems is that while 
there may be some way to describe a one-time interaction between processes 
there is no real construct to structure a series of reciprocal interactions 
between two parties \cite{HVM98}.

Session Types present a solution to the issue of structing communication-based 
software. In a similar way to how Object Oriented paradigms sought to solve 
the issues presented by large scale system written largely with spaghetti 
code, Session Types seek to restructure existing complex behaviours in a 
manner which is more lucid, readible and ultimately more easy to verify. Honda 
et al \cite{HVM98} present this in terms of simple concurrent primitives that 
build up a basic structuring method for communication-based concurrent 
programming. In terms of this project, we will initially be using simple 
communication primitives to check the program for such concurrency bugs as 
deadlocks. Sonic Pi appears a simple language to apply the theory of Session 
Types to but the dynamic and concurrent nature of the language give it some 
interesting communication patterns to formalise. Session Types 
are also designed in a language agnostic manner, meaning it will be possible to 
apply to Sonic Pi's Ruby-like syntax.

Session Types consist of the following key ideas:

\begin{itemize}
	\item A basic structural concept known as \emph{sessions}. These are designated 
	via \emph{channels}. The collection of session interactions is what constitutes a 
	program; those interactions are performed via the channels. As well as the 
	session, other concurrent programming contructs are provided: parallel 
	composition, name hiding, conditional and recursion. The combination of 
	recursion and sessions allow for the expression of an unbounded thread of 
	interaction as a single abstraction unit.

	\item Three basic communication primitives that all other structures will be 
	built from: \emph{value passing} - standard synchronised message-passing, 
	\emph{label branching} - purified method invocation, devoid of value passing - and 
	\emph{delgation} - the passing of a channel to another process. Alongside 
	sessions, these allow for complex communication structures to be defined and 
	described with clarity.

% flesh out detail and remark that the theory is not outlined here as
% there is no focus on the implementation of this in this project
% it would be good to extend the project to include a formal typing system
	\item Finally there is a basic type discipline for the communication primitives. 
	Without this there would be no way to guarantee the typability of a program, 
	ensuring that two communicating processes always have compatible patterns of 
	communication. It is the incompatibility of interaction patterns that is one 
	of the main reasons for bugs in commmunication-based programming\footnote{We choose not to present the full formalisation of the type system 
	in this report as the focus is much more on the construction of the 
	communication protocols than on their underlying types. Interested readers
	can read on this futher in such papers as \cite{HVM98} and \cite{HYC08}.}.
\end{itemize}

\subsubsection{Multi-Party Session Types}
Multi-Party Asynchronous Session Types are a class of behavourial types
specifically targeted at describing protocols in distribute systems
based on asynchrounous communication\cite{CCPY15}. As described previously,
communication interactions are intended to occur within the scope of many
private channels following strict protocols which we have labelled as
\emph{sessions}. In it's simplest form this takes place between just two
peers, hence our previous focus on ``binary'' session types (also known as
``dyadic''). In practice, a session can involve a variable number of peers
and so we extend the concept of session types and communication protocol
descriptions to involve the idea of \emph{multi-party} session types.

Multi-party session types have extended binary session types in a manner
that retains the intuitive natures of the syntax of the interactions. In
binary session types this came from the inherant notion of ``duality'' in
the interactions; a notion that is no longer effective in multi-party 
sessions as the whole conversation between processes cannnot be constructed
from a single behaviour. To handle this, multi-party session types
introduces the concept of a global type; an abstraction of the global
scenario, whereby we can construct the local types of each process and
again ensure the composability of the interactions \cite{HYC08}. 
\newpage

\section{Related Work}
\thispagestyle{empty}
%make more of a subsection thing? more discussion about tools?
This section discusses much of the work discovered during the intial research 
of the project. There are many interesting live programming environments in
the world, with each having an interesting approach to the technical 
challenges surrounding the issue of temporal semantics. We briefly discuss 
some languages/environments that have made use of the Session Types structures 
as this gives a clearer picture of their use and 
viability within the real world. It is worth noting that, to date, there is no 
apparent system which seeks to apply the Session Type protocol to a live 
programming paradigm and their associated concurrency features.

As noted by Rorhruber \cite{BMNR14}, there have been many publications and 
discussions relating to alternative approaches for temporal semantics and 
timing within Live Programming. There is much to be said about choosing 
between explicit and implicit representation of time as well as between the 
description of time using either internal or external state.

The language Tidal \cite{McL13} uses an interesting formalisation of cyclic 
time. Whilst drawing a continuing analogy with the act of knitting, McLean 
describes the DSL for musical pattern embedded in the pure functional language 
Haskell. Tidal represents music as a pure function, enabling the mapping of 
the single dimension of time into multidimensional music, making full use of 
iterative language to formalise cyclic time using both analogue and digital 
pattern.

Impromptu is a much fuller system, able to produce both audio and visual 
outputs in the context of Live Programming \cite{SG10}. It uses ``temporal 
recursion'' as a style of time-driven, discrete-event concurrency with real-
time interrupt scheduling. This recursion acts as an extension to the already 
existing support for real-time execution of arbitrary code blocks, with the 
real-time scheduler being responsible for the execution of the blocks in the 
correct ordering. Impromptu is designed to provide a reactive system with 
timing accuracy and precision based on constraints of human perceptions. Its 
choice of asynchronous concurrency allows a flexible architecture wherein 
Impromptu's co-operative concurrency model leaves the programmer responsible 
for time keeping and meeting real-time deadlines.

One of the closest other languages to Sonic Pi V2.0 is the language 
ChucK \cite{WC03}. ChucK is a strongly
-typed, imperative programming language whose syntax and semantics are 
governed by its flexible type system. The strength of the language lies in its 
(multiply) overloaded \texttt{=>} operator and its support of such features as 
dynamic control rates and strong concurrency principles. The similarity between 
this operator and the use of \texttt{sleep} to advance time make this language 
a very interesting source of comparison. The ordering of a 
program is captured naturally by the logic of the operator and ChucK allows 
the programmer, composer and performer to write truly concurrent code using 
the framework of the timing semantic (as controlled by this overloaded 
operator and a few select timing keywords). The manner in which ChucK advances 
time allows a level of granularity that makes it a stronger system than Sonic 
Pi in terms of musical performance but it is much less useable as a first 
programming language.

The timing effects of ChucK and the inherent expectation within music remind 
us that we must be able to speak clearly about the location of events in time. 
Therefore, any musical programming language must prove some form of time 
semantics, even if only informally. As previously mentioned with Sonic Pi 
V1.0, in the context of Live Programming this consideration extends to both 
situations where the code runs too late and where the code runs too early. An 
overlap between execution and creation time is a value of broader concern in 
software engineering, as noted by the Glitch system \cite{ME14}. This system 
allows the user to adjust notional execution time relative to a point in the 
source code editing environment. It proposes the idea that programming 
languages should address state update order by abstracting away from the 
computer's existing model of time; i.e. they should manage time in a way that 
draws analogy with memory management. 

Live Programming is not purely limited to musical languages and has similar 
applications is such areas as Logic, Dataflow and Artificial Intelligence in
terms of temporal reasoning. We do not list any such examples here as this is
not within the scope of this project\footnote{Interested readers are encouraged
to read \cite{AOB14} as they do detail some related lanuages in their closing 
sections.}.

As a demonstration for the potential of the application of Session Types 
within applications we briefly mention the protocol language Scribble 
\cite{HMBCY11}, whose protocols are very clearly defined using this theory. 
This has come about through the recognised need for widescale structing of 
protocols in light of the increasing amount of communication-based networking. 
Also of interest is Pabble \cite{NY14}: a system based off of the Scribble design 
and implemented using multiparty session types; a further extension to the 
binary session types discussed earlier.
\newpage

\section{Design \& Technical Implementation}
\thispagestyle{empty}
This chapter of the report focuses on the high level design of the project.
The sections are split into two core parts, one 
focusing on the design and implementation of the temporal behaviour, and
the second focused on the session types and associated graph structure.

The temporal behaviour of Sonic Pi is formalised through an effects system, also 
referred to as an abstract representation of time. We occassionally refer to this 
as the timing effects system of Sonic Pi, meaning the features implemented to 
handle reasoning about timing. We first describe this formal abstraction and then 
relate it to the approach of our implementation, which we present alongside the 
class structure of the project. 

Similarly in the second section we present the formal definitions of session types 
and then relate the theory to our implementation. The implementation is again 
presented alongside the class structure of this section.

We finish the chapter with a short focus on how the library is finally integrated
into the Sonic Pi IDE\footnote{As found at https://github.com/samaaron/sonic-pi}.
% \newpage

\subsection{Pi Time}
Before being able to process the source to analysis the virtual time, we must parse 
the source into a suitable data structure. A 
natural approach to this mimics compiler technology: we define some
grammar by which to parse our programs, and then build some Abstract Syntax Tree 
to store all of the program information. Once this 
is built successfully we are able to traverse the tree as often as we require 
to build the trace of the program, described in more detail in Section \ref{pTrace}.

Whilst ruby has been valued in Sonic Pi for its hugely forgiving
syntax in a classroom setting, this feature increases the difficulty of parsing 
it. The ability to write function calls either with or without parentheses 
or omit semi-colons between statements increases the complexity of the ruby 
grammar to the degree it was infeasible to write such a parser specifically 
for Sonic-Pi. However, given Sonic-Pi's ruby-like nature, we elected to use 
the open-source ruby-parser\footnote{As found at https://github.com/whitequark/parser}
for its simplicity and ease of set-up. It is relatively lightweight whilst 
still providing useful statistics such as line numbers, statement counts and
column numbers. 
\newpage

\subsubsection{Building the AST} \label{build}

\begin{figure}[h!]
	\centering
	\includegraphics[width=0.85\textwidth]{images/program.jpg}
	\caption{A Simple Program AST} \label{program}
\end{figure}

The ruby-parser builds the AST with a rose like structure which we continue to 
utilise throughout our project as it provides many useful node types. The most 
interesting nodes are \texttt{block} and \texttt{begin}; \texttt{block} for its 
use in analysis and \texttt{begin} for how it manages statements. 

\begin{minipage}{\textwidth}
	\begin{lstlisting}[style = sonicpi]
            loop do
                play 60
                sleep 1
                loop do
                    sleep 1
                end
            end
	\end{lstlisting}
	\captionof{lstlisting}{Simple Loop} \label{astLoop}
\end{minipage}

Figure \ref{program} captures the AST of Snippet \ref{astLoop}, shown above. 
Loops and function definitions are denoted by \texttt{block} types. This \texttt{block}
can act as a root node, enabling us to work with each loop and function as separate 
trees during analysis, compounding the results at the end. The full structure of a 
\texttt{block} is shown in Figure \ref{blocks}. A \texttt{block} will 
always have three children: the first names the type of block, the second contains 
the arguments given, and the third is the statement list. This statement list is 
either marked with \texttt{begin} or contains a single statement node. Figure 
\ref{program} demonstrates both of these examples. The outermost \texttt{loop} 
contains three statements, so the AST marks the start of that statement list with 
\texttt{begin} which holds all the following statements as children. The innermost 
\texttt{loop} holds only a single statement, so \texttt{begin} is omitted.

% block node diagrams
\begin{figure}[h!]
	\centering
	\includegraphics[width=\textwidth]{images/Block.jpg}
	\caption{Block Node Structure} \label{blocks}
\end{figure}

This rose-like structure means that our tree is searched in a breadth-first manner. 
Take the following tree diagram, which demonstrates how node numbering is managed. 
When searching for index 8, breadth-first search means that we will search each 
child, 1 then 4 then 10, before visiting any of their subsequent children. By 
testing that the index is greater than the current index but less than the next child, 
we reach our target quickly.
\\
\begin{lstlisting}
                                 0
                           /     |     \
                          1      4      10
                         / \    / \    / 
                        2   3  5   8  11 
                              / \   \
                             6   7   9
\end{lstlisting} 

This child structure makes it simple to detect sequential and nested \texttt{block}s. 
As shown in Figure \ref{blockdetect}, sequential \texttt{block}s will always share a 
common parent (\texttt{begin}). Otherwise, as with the nesting in Figure \ref{program}, 
encountering a new \texttt{block} with a differing parent suggests nesting. This is 
leveraged later when we detail the timing analysis in \ref{pTrace}.

% block detection diagrams
\begin{figure}[h!]
	\centering
	\includegraphics[width=\textwidth]{images/BlkDetection.jpg}
	\caption{Block Detection} \label{blockdetect}
\end{figure}
\vspace{15pt}
In transferring this data to our own C++ application we simplify the node count of 
our AST by wrapping certain nodes into 
tighter definitions. For example, the ruby-parser outputs an integer as a node 
of type \texttt{int} with a child holding the actual integer value. We fold these 
definitions into a single node of type \texttt{IntNode} with held the integer value 
as a member field. \texttt{float} and \texttt{symbol} nodes act in a similar
fashion, as demonstrated with \ref{symwrap}.

% diagramtic example of a sym node
\begin{figure}[h!]
	\centering
	\includegraphics[width=0.85\textwidth]{images/SymInt.jpg}
	\caption{Node Folding} \label{symwrap}
\end{figure}


\subsubsection{Formalising Abstract Time} \label{abstractTime}
Before writing the temporal behaviour, we provide an abstract definitions of time 
for Sonic Pi to work with. This representation can be thought of as a `time system' 
and enables us to define a concept of `time safety'. By proving the semantics sound, 
with respect to this system \cite{AOB14}, we are able to reject those programs with 
bad temporal behaviour. This static verification enables developers to reason about 
their program's timing intuitively.

The syntax of Sonic Pi V2.0 can be outlined simply.
\\
\begin{lstlisting}
	P ::= P; S | $\emptyset$
	S ::= E | $v$ = E | def $f$ : P | $f$
	E ::= sleep $\mathbb{R}_{{\geqslant}0}$ | A$^{i}$ | $v$ | if $v$ then P else P
\end{lstlisting}

\texttt{P} represents a full program, 
\texttt{S} are statements (either self-contained expressions or pure bindings 
to variables, $v$), and \texttt{E} defines expressions. \texttt{A$^{i}$} is used to 
refer to all those operations that will not advance time, e.g. the 
\texttt{play} operation. In defining the syntax this way we are able to abstract 
over the operations that do not modify virtual time.

We extend the definition given in \cite{AOB14} with $f$ denoting a string used for 
either function definitions or function calls, and the rule for conditionals. 
\texttt{def $f$ : P} means to say a function with name $f$ is defined with program 
\texttt{P}; $f$ is a function call to that definition.

In distinguishing the real time elapsed by the program and the virtual time elapsed 
we write [\texttt{P}]$_{t}$ for the former and [\texttt{P}]$_{v}$ for the latter. 
Both of these abstract functions return time values and so are positive, 
real-number values.

\begin{blockquote}
	\textbf{Definition 1:} \\ 
	Virtual time is specified for Sonic Pi using the following cases:
	\\
	\begin{lstlisting}
     $\hphantom{..}$   [P; $v$ = E]$_{v}$ = [P]$_{v}$ + [E]$_{v}$       # Statement Sequences
               [$\emptyset$]$_{v}$ = 0               # Empty Program
         [sleep t]$_{v}$ = t              # Sleep Statement
            $\hphantom{.....}$   [$v$]$_{v}$ = 0               # Variable
            $\hphantom{...}$ [A$^{i}$]$_{v}$ = 0                # Other Statements
	\end{lstlisting}

	We extend the cases here (as presented from \cite{AOB14}) with functions, loops
	and conditionals:
	\\
	\begin{lstlisting}
            [$f$]$_{v}$ = [def $f$ : P]$_{v}$ = [P]$_{v}$    # Function Call

	             [loop P]$_{v}$ = [P]$_{v}^{*}$   # Loops
	  $\hphantom{v}$       [loop P; P']$_{v}$ = [P]$_{v}^{*}$ + [P']$_{v}$
	                       = [P]$_{v}^{*}$


                                      # Conditional
        [if $v$ then P$_{1}$ else P$_{2}$]$_{v}$ = [$v$]$_{v}$ + ([P$_{1}$] $\emph{max}$ [P$_{2}$])
                               = [P$_{1}$] $\emph{max}$ [P$_{2}$]
	\end{lstlisting}

	The * operation marks a virtual time of infinite length. It is defined as a 
	subsuming operation. 
	\\
	\begin{lstlisting}
	$\mathbb{N}$ $\cup$ {$\infty$}            $\infty$++ = $\infty$
	*: $\mathbb{N}$ $\rightarrow$ $\mathbb{N}^{*}$           ++$\infty$ = $\infty$
	\end{lstlisting}
\end{blockquote}

\begin{figure}[h!]
	\centering
	\includegraphics[width=0.9\textwidth]{images/timing.jpg}
	\caption{Tree UML} \label{tree}
\end{figure}

\subsubsection{\texttt{pTrace}} \label{pTrace}



In building this \texttt{pTrace} we are mapping our existing tree structure to a 
flat list structure. We could have simply stored the timing information into the 
nodes themselves but direct index access to the informations proves preferential 
when integrating data into the IDE. As shown in Figure \ref{tree}, each 
\texttt{NodeTree} holds a reference to a \texttt{pTrace}, the data structure we 
use to store all of the important analysis results for the current program. The main 
structure of interest within this \texttt{pTrace} is the vector of statements 
allowing us to utilise the \texttt{[]}-operator when examining specific statements.

Each index contains a struct of useful data such as the node index of the statement,
contributive virtual time (referred to here as \texttt{conVT} and indicitive of a 
sleep statement or function call), cumulative virtual time (referred to as 
\texttt{conVT} and used to know what the total virtual time is at each statement), 
and other fields relating to function calls. When the statement is a \texttt{sleep t} 
statement, it marks that the contributing virtual time, \texttt{conVT}, is of length 
\texttt{t}. Most other statements, aside from function calls, are given a 
\texttt{conVT} of zero. Function calls contribute the amount of virtual time they 
take, the values of which are stored in a map within \texttt{pTrace}.

The analysis processes the tree into \texttt{pTrace} using the specification 
presented in Section \ref{abstractTime}. This happens in two `passes' by way of 
interprocedural analysis. The `first pass' refers to a full traversal of the AST, 
building a \texttt{pTrace} of the correct length and marking each \texttt{sleep} 
statement, function call and conditional where it occurs. The `second pass' refers 
to an iteration over the \texttt{pTrace} where we use the data collected to fill 
in the results of function calls and process conditional brances. 

\paragraph{Loops}
To start with a simple example, Snippet \ref{simpleLoop} shows the code for the
following simple trace.

\begin{multicols}{2}
	\begin{minipage}{0.4\textwidth}
		\begin{lstlisting}[style = sonicpi]

    loop do    # 1
      play 60  # 2
      sleep 2  # 3
      play 64  # 4
    end

		\end{lstlisting}
		\captionof{lstlisting}{Simple Loop} \label{simpleLoop}
	\end{minipage} \hspace{25pt}
	\begin{minipage}{0.4\textwidth}
	\begin{lstlisting}
== Trace
[0] -                 
[1] conVT: 0, cumVT: 0, 
    isCall: false, inFunc: false
[2] conVT: 0, cumVT: 0, 
    isCall: false, inFunc: false
[3] conVT: 2, cumVT: 2, 
    isCall: false, inFunc: false
[4] conVt: 0, cumVT: 2, 
    isCall: false, inFunc: false
	\end{lstlisting}
	\end{minipage}

\end{multicols}

The information for each statement is mapped to the index of the same statement 
count. Index 0 is never filled as this is always assigned to the root node, and 
because it is natural for novice programmers to index from 1 rather than 0. Given 
this, we often skip handling the first index of our trace because it is always a 
zero slot \footnote{This is based on the fact a program is normally a sequence of 
statements or a loop. A check has been put in place should certain constructs be 
processed in isolation. This is addressed specifically in Section \ref{evalCond}.}.
Snippet \ref{simpleLoop} demonstrates quite neatly how simple it can be to collect 
the cumulative virtual time as we iterate through the program.

In Section \ref{abstractTime} we define a loop's virtual time as being infinite. 
This is, however, not a useful result to return to the developer. Instead, each 
unconstrained loop is traversed once so that the value of one iteration is known. 
All the information returned to the developer is based on running one iteration 
of the whole source as this is a useful metric for the developer to reason against. 
This is valid within our specification as we have defined * as mapping natural 
numbers to infinity. \texttt{[loop P]$_{v}$} in this case equals \texttt{[2]$_{v}^{*}$} 
which equals \texttt{$\infty$}.

\paragraph{Functions}
Snippet \ref{doubleFunc} gives a program where functions have been defined before 
and after their use. This is a valid ruby program and will compile and run within 
the Sonic Pi IDE. However, Sonic Pi's current scoping system mean that this program 
will fail during runtime on reaching statement 4\footnote{The symbol called is not 
yet defined within Sonic Pi's scope so the symbol is unrecognised and the thread 
crashes.}. This kind of error is outside the scope of the project to provide an error 
report for so we do not save the developer in this instance. Instead we handle it, 
as it compiles within the environment and is sent for evaluation by the underlying 
servers. Thankfully, our interprocedural analysis allows us to handle this case for 
free, so we use it here as a strong example of function calls and their temporal 
behaviour. 

\begin{minipage}{\textwidth}
	\begin{lstlisting}[style = sonicpi]
        define :first do       # 1
          play 60              # 2
        end
        
        first                  # 3
        second                 # 4
        play 60                # 5

        define :second do      # 6
          sleep 1              # 7
        end
	\end{lstlisting}
	\captionof{lstlisting}{Surrounding Functions} \label{doubleFunc}
\end{minipage}
\\
\begin{lstlisting}
    == Trace
    [0] -
    [1] conVT: 0, cumVT: 0, isCall: false, inFunc: true
    [2] conVT: 0, cumVT: 0, isCall: false, inFunc: true
    [3] conVT: 0, cumVT: 0, isCall: true,  inFunc: false
    [4] conVT: 1, cumVT: 1, isCall: true,  inFunc: false
    [5] conVT: 0, cumVT: 1, isCall: false, inFunc: false
    [6] conVT: 0, cumVT: 0, isCall: false, inFunc: true
    [7] conVT: 1, cumVT: 1, isCall: false, inFunc: true
\end{lstlisting}

The issues that Snippet \ref{doubleFunc} identifies is how to insert the correct
amount of contributing virtual time into the trace when you might not have 
encountered the function being called. The simple way to handle this is by processing
the data in two passes. On the first pass of a program we collect information
such as which lines are function calls and what all of our function definitions
and times are. It is not printed in these traces but one of the extra things that
each statement index holds is, if the current line is a function call, which function
does it call.

The trace after the first pass of this program looks as follows:
\\
\begin{lstlisting}
    == Trace
    [0] -
    [1] conVT:  0, cumVT:  0, isCall: false, inFunc: true
    [2] conVT:  0, cumVT:  0, isCall: false, inFunc: true
    [3] conVT: -1, cumVT: -1, isCall: true,  inFunc: false
    [4] conVT: -1, cumVT: -1, isCall: true,  inFunc: false
    [5] conVT:  0, cumVT: -2, isCall: false, inFunc: false
    [6] conVT:  0, cumVT:  0, isCall: false, inFunc: true
    [7] conVT:  1, cumVT:  1, isCall: false, inFunc: true
\end{lstlisting}

Once this is processed it is again a simple case of adding up the cumulative virtual
time as we traverse the trace. \texttt{-1} is reserved as a keycode in tandem with
the \texttt{isCall} marker; if a function has \texttt{conVT = -1} it means this
is a function call and will be processed in a later pass. It might seem like the 
keycode is useless in this case, but as the virtual time keycode is consumed during 
trace processing, it is useful to maintain the other boolean flag so that function 
calls are more easily identified afterwards.

The key idea is not to add together times from different blocks. The 
\texttt{inFunc} note helps with this, as we can detect when we are in a new block 
of the trace by what value that has. Notice that, in the above trace, index 6 
has a fresh \texttt{cumVT} value when compared with index 5. The issue here is that 
the \texttt{inFunc} marker is still too simplistic. This handles sequential functions 
as each statement in \texttt{pTrace} holds information about which function it is 
part of when \texttt{inFunc} is set. Otherwise the model would be too simple and 
function times would run into each other. This is detailed more in Section 
\ref{evalFunc}.

\paragraph{Conditionals}
Similar to functions, we require a two pass approach to handle conditionals in the 
code. Take the following program code:

\begin{minipage}{\textwidth}
	\begin{lstlisting}[style = sonicpi]
        if true then      #1
            sleep 0.5     #2
        else
            sleep 1       #3
        end
	\end{lstlisting}
	\captionof{lstlisting}{Conditional} \label{conditional}
\end{minipage}

In practice it may be difficult to evaluate which branch of a conditional the
program is going to take. Instead the approach is to find the maximum virtual time 
of both branches and take the longest one, as described in Section \ref{abstractTime},
Definition 1. In terms of timing later in the program this could potentially throw 
developers on what the actual virtual time of the program will be if the 
\texttt{sleep} lengths differ by a sizeable margin. On the other hand this is a much 
safer model in terms of ensuring the `time saftey' of the program\footnote{The real 
running time of the code is guaranteed to be longer than the virtual time, as proved 
in \cite{AOB14}, Section 4.1.}.

The trace for Snippet \ref{conditional} is:
\\
\begin{lstlisting}
    == First Pass Trace
    [0] -
    [1] conVT:  -2, cumVT: -2,   isCall: false, inFunc: false
    [2] conVT: 0.5, cumVT: -1.5, isCall: false, inFunc: false
    [3] conVT:   1, cumVT: -0.5, isCall: false, inFunc: false

    == Second Pass Trace
    [0] -
    [1] conVT:  1, cumVT:    1, isCall: false, inFunc: false
    [2] conVT: -3, cumVT: -1.5, isCall: false, inFunc: false
    [3] conVT: -3, cumVT: -0.5, isCall: false, inFunc: false
\end{lstlisting}

In the first pass of the trace, we mark the location of the if with the keycode 
-2. After this we calculate the virtual time of each branch and store the index 
for the last state of each branch in the \texttt{IfNode} at the root of this 
expression block. Similar to how block detection works, the end of an if-branch 
is detected when the next index in question has the same parent index, or one 
of greater value (or the program has finished in the case the conditonal is at 
the end of the program). With this, the second pass is simply a case of, on reaching 
a keycode of -2, set the result of whichever branch was longest.

\paragraph{Dead Code}

When calculating the virtual time of the program it is quite important not to 
take dead code into account. Dead code can easily push the trace into incorrect 
evaluation and make it difficult for the developer to calculate the time in their 
program. As well as this, being able to print when code will not be reached in a 
program is incredibly useful information for a developer to have in such an 
environment. Dead code can also be easy to locate thanks to the form of the program.

\begin{minipage}{\textwidth}
	\begin{lstlisting}[style = sonicpi]
      loop do
        play 60
        sleep 1
      end

      loop do
        play 60
        sleep 1
      end
	\end{lstlisting}
	\captionof{lstlisting}{Dead Code} \label{dead}
\end{minipage}

We set \texttt{-3} to be the dead keycode and set all instances of \texttt{conVT} 
to this value when the statement is registered as dead. Here the second loop is 
obviously never going to be run, so we must remove this from consideration in our
trace. The trace makes a note of when it has seen a loop block and sets a boolean 
flag for dead code on hitting another loop block. Should the trace hit a 
\texttt{in\_thread} block or a \texttt{live\_loop} block then this flag is not set 
as these pieces of code can run in parallel.

This is captured in our specification by the subsumption of *. As infinity plus 
any value is still infinity, we can consider the subsequent program to have 
virtual time zero, which amounts to not running it.

\paragraph{Variable Sleep}

\begin{minipage}{\textwidth}
	\begin{lstlisting}[style = sonicpi]
      define :foo do |n|
        sleep n
      end
	\end{lstlisting}
	\captionof{lstlisting}{Variable Sleep} \label{variableSleep}
\end{minipage}

Snippet \ref{variableSleep} shows a common construct in Sonic Pi; the idea of 
functions allowing variable sleep amounts. This is not currently supported in this 
iteration of the project\footnote{This is owing to time constraints and feature prioritization.}, but the foundations for it are well-placed. Each 
\texttt{block} defines an \texttt{arg} child which lists the symbols for all 
arguments passed to it. Given this we can define a new keycode in our \texttt{pTrace}
and use the same style of interprocedural analysis as before. In the case of Snippet 
\ref{variableSleep}, the function can be defined in terms of its total variable time.
Then, when processing the function call itself, the numerical argument is taken and 
applied to the expression stored for the function\footnote{Argument symbols are 
stored as \texttt{lvar} type nodes, making expression detection and evaluation 
simple to manage.}. A full trace of this is given in Section \ref{evalFunc}.

\subsection{Session Pi} \label{sessionPi}
Session Types describe communication protocols via message passing. In Sonic Pi
the message originates at a \texttt{cue} statement and is `consumed' by the
\texttt{sync} statement. Given this, we elect to transform the AST used in
timing analysis into a directed graph to better represent the interactions. In this 
chapter, we first formalise the concepts of Session Types; we also set the 
context for Global Types and Multi-Party Session Types through a simple example and 
formalise the idea of \emph{Projection}. After this we describe the design of the 
graph, which details two core iterations of the graph. Finally, we relate 
this implementation to the theory presented in \ref{formalST}. We explain our 
approach to \emph{Global Inference} and present our own extensions to Multi-Party 
Session Types: the \texttt{\&\&}, and \texttt{||} types.

\subsubsection{Formal Session Types} \label{formalST}
Before writing the communication interactions of Sonic Pi, we provide a formalisation 
of Session Types. This defines a series of constructs that may be used to describe 
the types of communication a process may have and which other processes it may 
interact with. By defining and analysing these patterns we are able to verify which 
programs are safe from concurrency errors like deadlocks and reject those that have 
bad communication structure. This static verification enables developers to easily 
reason about the different interactions within their program.

As in Section \ref{abstractTime}, the syntax of Sonic Pi V2.0 is simple.
\\
\begin{lstlisting}
	P ::= P; S | $\emptyset$
	S ::= E  | $v$ = E
	E ::= sleep $\mathbb{R}_{{\geqslant}0}$ | cue $v$ | sync $v$ 
	    | A$^{i}$ | $v$
\end{lstlisting}

We expand the syntax to define \texttt{cue} and \texttt{sync} which operate on
some variable name, $v$. We maintain our definition of \texttt{sleep} in this syntax 
as it is a useful tool in our reasoning. \texttt{A$_{i}$} is thus the set of all 
operations that either do not advance time or do not consist of some communication
primitive. In this way we can abstract over the operations that do not contribute 
to communications.

This definition is lacking the syntax for function calls and conditionals as this 
analysis does not currently support them. These limitations are discussed in greater 
details towards the end of Section \ref{sessionPi}.

\begin{minipage}{\textwidth}
	\begin{lstlisting}[style = sonicpi]
           # P0                        # P1
           in_thread do                in_thread do 
             loop do                     loop do
               cue :B                      cue :A 
               sync :A                     sleep 1
               sleep 1                     sync :B
               play 63                     sleep 0.5
             end                         end
           end                         end
	\end{lstlisting}
	\captionof{lstlisting}{Forming Types} \label{formTypes}
\end{minipage}

For the following section, the base set of syntax we use to describe our session
types are as follows\footnote{Other sets used are \emph{constants}, 
\emph{expressions}, and \emph{process variables} but we do not need to consider 
these for the language constructs discussed in this report.}:

\begin{itemize}[noitemsep]
  \item[] \emph{names}: ranged over by \emph{a,b...}
  \item[] \emph{channels}: ranged over by \emph{k, k'...}
  \item[] \emph{variables}: ranged over by \emph{x, y...}
  \item[] \emph{labels}: ranged over by \emph{l,l'...}
\end{itemize}

The final set of interest is that of \emph{Processes}, ranged over 
by \texttt{P, Q...}. For this discussion we define the following aspects 
of the \emph{Process} grammar:
\\
\begin{lstlisting}
    P ::= k![$\tilde{e}$]; P                    data sending
        | k?($\tilde{x}$) in P                  data reception
        | def D in P                 recursion
\end{lstlisting}
\begin{lstlisting}
    D ::= X$_1$($\tilde{x}_1\tilde{k}_1$) = P1 and ...           declaration for recursion
                and X$_n$($\tilde{x}_n\tilde{k}_n$) = P$_n$ 
\end{lstlisting}

% Parenthesis denote binders which bind the corresponding free occurrences. The sets 
% of free names/channels/variables of, for example process \texttt{P}, are written as 
% \texttt{fn(P)}, \texttt{fc(P)}, and \texttt{fv(P)} respectively. Processes without 
% free variables or free channels are called programs.
We maintain this definition of recursion as it is useful to be able to apply it to 
the loops in our processes. Similarly to the way we defined an infinity operator in 
Section \ref{abstractTime}, this allows us to only consider the first iteration of a 
loop when handling session types. For the full performance the session types can 
unfold as many times as you require, but the developer only needs to see that 
type information once.

With binary session types defined we may now formalise the idea of a dual type. The 
dual relation defines the link between two endpoints of communication between two 
participants in binary sessions. The endpoint of communication for each participant 
can be thought of as the next message to consume. In this grammar we have one dual 
relation that we are concerned with. 

\begin{blockquote}
	\texttt{! = dual(?)}
\end{blockquote}

This is an important statement as it is what we use to define the way our protocols 
interact with each other. In addition to these binary type relations, this project 
uses the multi-party session typing paradigm, which presents a slight
improvement to this syntax, allowing our communication to occur between more
than two \emph{participants}. 

\emph{Participants} are the set ranged over by 
\emph{p, q,...}
\\
\begin{lstlisting}
    P ::= c!<p,e>.P             value sending
        | c?(p,x).P             value reception
\end{lstlisting}

With this we can now state which of our participants we are sending our message
to and on which channel. With this we are now able to begin typing our programs.

\paragraph{Global Types}
To begin illustrating global types by example, we present a likely scenario to occur 
in a piece of music. First drum, \texttt{D1}, sends a `cue' to the melody, 
\texttt{M1}, to start the song in sync. In responce \texttt{M1} sends a `cue' to 
both \texttt{D1} and a half-drum, \texttt{D2}.  After some time has passed, 
\texttt{D2} may have fallen behind time slightly, so \texttt{D1} sends another 
message to sync these two beats together. After this \texttt{D2} can `cue' a waiting 
riff, \texttt{M2}, to decorate the music. Decomposing this into several binary 
session types could be logically very difficult, and most likely we would lose some 
of the essential sequencing information in the process. Instead, let us transform 
this scenario into the following global type.
\\
\begin{lstlisting}
      D1 $\rightarrow$ M1 . M1 $\rightarrow$ D1 . M1 $\rightarrow$ D2 .
      D1 $\rightarrow$ D2 . D2 $\rightarrow$ M2 
\end{lstlisting}

This example shows how simply the communication structure of a program, using 
multiple processes, can be described using Multi-Party Session Types. 
In our analysis, a message can defined a passing from process \texttt{P0} to 
process \texttt{P1}, written \texttt{P0->P1}, if the types in processes \texttt{P0} 
and \texttt{P1} dual each other. The global types describe the various actions of a 
participant \texttt{p} sending either a value of a given Sort, a channel of a given 
Type, or some label to the partcipant \texttt{q} and then interaction continues 
according to the following global type \texttt{G}. The grammar we are interested in 
here is as follows:
\\
\begin{lstlisting}
          G ::= p $\rightarrow$ q : $\langle$S$\rangle$.G           # Value Exchange
              | p $\rightarrow$ q : $\langle$T$\rangle$.G           # Channel Exchange
              | $\texttt{\textbf{t}}$ | end               # Recursion/End
      
          S ::= bool | ... | G       # Sorts
          U ::= S | T                # Exchange Types
      
          T ::= !$\langle$p, S$\rangle$.T              # Send Value
              | !$\langle$p, T$\rangle$.T              # Send Channel
              | ?(p, U).T            # Receive
              | $\texttt{\textbf{t}}$ | end               # Recursion/End
\end{lstlisting}

\paragraph{Projection}
The global type is the description of the program as the whole. It shows which 
participants communication with which and what data they pass between each other. 
The decomposed local type describes the interaction that specific process is aware 
of.

\begin{blockquote}
  \textbf{Definition 2:} \emph{The projection of a global type \texttt{G} onto 
  a particiapnt \texttt{q} is defined by induction on \texttt{G}}:
\end{blockquote}
\begin{align*}
  (\texttt{p} \rightarrow \texttt{p'}:\texttt{<U>}.\texttt{$G$'}) \upharpoonright \texttt{q = }  
  \begin{cases}
\hspace{5pt} !\langle \texttt{p',U} \rangle (\texttt{$G$'}\upharpoonright \texttt{q}) \hspace{22pt}if \hspace{5pt} \texttt{q = p} \\
\hspace{5pt} ?(\texttt{p',U}) (\texttt{$G$'}\upharpoonright \texttt{q}) \hspace{20pt}if \hspace{5pt} \texttt{q = p'} \\
\hspace{5pt} \texttt{$G'$}\upharpoonright \texttt{q} \hspace{65pt} otherwise
  \end{cases}
\end{align*}
\begin{align*}
  \texttt{t} \upharpoonright \texttt{q = t} 
  \hspace{40pt} \texttt{end} \upharpoonright \texttt{q = end}
\end{align*}

This approach is not so feasible in this instance as we have already been 
presented with the full source. Instead we are required to build the global
type from the individual processes, the aim being to verify
the communication of the program. This consists largely of 
matching operations on the channels they operate on. Whilst notably more
difficult, this approach is also, thankfully, proveably decidable. There is
no situation where any algorithmic approach to the problem will not terminate
with some answer. In the cases where our program's global type is not
typeable, this suggests some communication error within the provided program.

\begin{figure}[h!]
	\centering
	\includegraphics[width=0.77\textwidth]{images/session.jpg}
	\caption{Graph UML} \label{graphUML}
\end{figure}
\newpage

\subsubsection{Planning the Graph} \label{planningGraph}
As shown by the Figure \ref{graphUML}, each \texttt{Graph} is constructed as
a list of \texttt{SubGraph}s where each \texttt{SubGraph} represents
one process within the program. This structure is shown again more explicitly in 
Figure \ref{graphStructure}, which also shows the various types of arc that this 
analysis graph makes use of.

\begin{figure}[h!]
	\centering
	\includegraphics[width=\textwidth]{images/GSG.jpg}
	\caption{Graph Structure} \label{graphStructure}
\end{figure}

The idea behind this is to maintain the chronology of the nodes in each process and 
to enable easy analysis of the program on a per-process level. In Sonic Pi there is 
a \texttt{cue} operation, which sends out an asynchronous message to all processes. 
To complement this is a \texttt{sync} operation, a blocking operation that causes 
the process to wait to receive a message. Immediately we draw the parallel between 
these actions and the sending/receiving messages that Session Types describe. 
Linking back to the dual relation, \texttt{cue} acts as our \texttt{!} type and 
\texttt{sync} acts as our \texttt{?} type. \texttt{SubGraph} handles the construction 
of local types while the \texttt{Graph} class handles the global type.

Our approach to building the global type with this graph is simple. We describe it 
here with pseudo code:

\begin{blockquote}
	\begin{lstlisting}
  syncStore  <- new
  prevTokens <- new   # initial attempt with allowing sub-typing

  while every process has tokens
  |   or every non-empty process is not in syncStore:
  |   currTokens <- new
  |   for each process:
  |   |   # sync is blocking
  |   |   # so as long as a process has a token in sync
  |   |   # we can't consume it
  |   |   if syncStore.members.name != process.name
  |   |   |   currTokens += process.top
  |
  |
  |   check if currTokens dual each other:
  |   |   build type; remove from token list
  |	
  |   check if currTokens dual any in syncStore
  |   |   build type; remove from respective store
  |   
  |   check if currTokens dual any in prevTokens
  |   |   build type; remove from respective store
  |
  |   syncStore += currTokens.syncTokens
	\end{lstlisting}
\end{blockquote}

\begin{figure}[h!]
	\centering
	\includegraphics[width=\textwidth]{images/GraphOneFix.jpg}
	\caption{Node Chronology} \label{nodegraphone}
\end{figure}

The reason we process tokens in this way is demonstrated in Figure 
\ref{nodegraphone}. A Sonic Pi program will always increment its virtual 
time as we run through a program. \texttt{SubGraph}s are written in the order 
they are defined in the source by line number. Given this, within a 
\texttt{SubGraph} if \texttt{n1.index < n2.index}, \texttt{n1} has resolved 
first. We will refer to this as the `horizontal' relationship 
between nodes. In two concurrent \texttt{SubGraph}s, \texttt{n.index < m.index}, 
and \texttt{n} will be resolved first by the interpreter, but they may have the 
same virtual time within the program. Given this they are compared to each other as 
well as those \texttt{sync} tokens processed previously. We will refer to this as 
the `vertical' relationship between nodes and is demonstrated in Figure 
\ref{nodegraphone} by the blue bar linking the different groups together.

% diagram of internal graphs here, showing that nodes in these graphs 
% are chronological


There is a small limitation in this design as it will allow certain untypable
programs to be typed.

Take the two code snippets below. Snippet \ref{typed} is typeable and Snippet 
\ref{untyped} is not, yet both will produce this sesstion type:
\\
\begin{lstlisting}
  P0 := foo:(P0P1)!.bar:(P1P0)?
  P1 := bar:(P1P0)!.foo:(P0P1)?
\end{lstlisting}

The analysis falls apart here because the basic method of iterating over each local 
type does not translate well when handling sub-typing.

\begin{multicols}{2} 
	\begin{minipage}{0.4\textwidth}
		\begin{lstlisting}[style = sonicpi]
live_loop :foo do    # P0
    play :e4, release: 0.5
    sleep 0.5
    sync :bar
end

live_loop :bar do    # P1
    sample :bd_haus
    sleep 1
    sync :foo
end
		\end{lstlisting}
		\captionof{lstlisting}{Untypable} \label{untyped}
	\end{minipage} \hspace{35pt}
	\begin{minipage}{0.4\textwidth}
		\begin{lstlisting}[style = sonicpi]
live_loop :foo do    # P0
    sync :bar
    play :e4, release: 0.5
    sleep 0.5
end

live_loop :bar do    # P1
    sync :foo
    sample :bd_haus
    sleep 1
end
		\end{lstlisting}
		\captionof{lstlisting}{Typable} \label{typed}
	\end{minipage}
\end{multicols}
\newpage

\paragraph{Sub-Typing}
\begin{multicols}{2}
Under the initial algorithm, both \texttt{cue} statements in Snippet \ref{subType} 
would be processed and then discarded, as \texttt{cue} statements do not persist in 
the statement store. However, these statements all execute at the same virtual time 
in the program; both \texttt{cue}s will trigger both \texttt{sync}s in the Sonic 
Pi environment. Storing the previous round of tokens has also already been shown to 
be inaccurate.
\\
\begin{lstlisting}
  P0 := B:(P0P1)!.A:(P1P0)?
  P1 := A:(P0P1)!.B:(P0P1)?
\end{lstlisting}

In standard binary session types this is inaccurate:

$\hphantom{tabs}$ \texttt{B:(P0P1)!.A:(P1P0)?} 
\\ $\hphantom{tabular} \neq$ \texttt{dual(A:(P0P1)!.B:(P0P1)?)}

The accurate type for the second process would be \texttt{B:(P0P1)?.A:(P0P1)!}, 
(or one could keep \texttt{P1} unchanged and swap the terms in the other process). 
By introducing the concept of time progression, we are able to handle 
this typing.

	\begin{minipage}{0.5\textwidth}

		\begin{minipage}{\textwidth}
			\begin{lstlisting}[style = sonicpi]
	# P0 
	in_thread do
	  loop do   
	    cue :B  
	    sync :A 
	    play 60 
	    sleep 0.5 
	  end 
	end           


	# P1
	in_thread do
	  loop do
	    cue :A
	    sync :B
	    play 64
	    sleep 0.5
	  end
	end
			\end{lstlisting}
			\captionof{lstlisting}{Sub-Typing \texttt{cue}} \label{subType}
		\end{minipage}

	\end{minipage}
\end{multicols}

Our implementation utilises sub-typing the \texttt{cue} type. \\
The formal outline for this feature is:
\\
\begin{lstlisting}
     A!.P <: P.A!
\end{lstlisting}

Given this we may say that:
\\
\begin{lstlisting}
     A:(P0P1)!.B:(P0P1)? <: B:(P0P1)?.A:(P0P1)!
     B:(P0P1)!.A:(P1P0)? = dual(B:(P0P1)?.A:(P0P1)!)

    $\therefore$ B:(P0P1)!.A:(P1P0)? $\cong$ dual(A:(P0P1)!.B:(P0P1)?)
\end{lstlisting}

Sub-typing is a useful relation to define within session types, see 
\cite{HYC08, MY15, MYH09}, and is based on the notion that correctness will 
always be maintained so long as input order is not disrupted in the message 
queue. Here, our subtype is based on operations occuring at the same `virtual 
time', so we may consider operation ordering to be maintained in this instance. 


\paragraph{Keeping Time}
To handle this feature we introduce a blank \texttt{GraphNode} into the graph 
structure described in Section \ref{planningGraph}. This node represents when virtual 
time has incremented in the program, helping us identify which communication 
operations interact with which.

This does not break the order relation
described previously but it does change the grouping of the `vertical' 
relationship of the nodes. Rather than mapping each column as one relation, with
the total set of relationship being equal to the maximum number of types in
any one \texttt{SubGraph}, each `vertical' relation can be defined as the
set of nodes between each \emph{time} node.

% updated graph diagram here
% highlights each vertical relation group
\begin{figure}[h!]
	\centering
	\includegraphics[width=\textwidth]{images/GraphTwo.jpg}
	\caption{Fixed Node Chronology} \label{graphTwo}
\end{figure}

Refering back to Snippet \ref{typed}, the program graph constructed now takes
the form:
\\
\begin{lstlisting}
  P0 :=  foo:(P0P1)!.bar:(P1P0)?.time
  P1 :=  bar:(P1P0)!.foo:(P0P1)?.time
\end{lstlisting}

More importantly the program from Snippet \ref{untyped} will look as follows:
\\
\begin{lstlisting}
  foo := foo:(P0P1)!.time.bar:(P1P0)?
  bar := bar:(P1P0)!.time.foo:(P0P1)?
\end{lstlisting}

When analysing these local types, the only change to the pseudo code presented before 
is how we consume tokens at the start of each loop. Instead of only taking the 
top token from each process, we keep collecting tokens until we either hit a 
\texttt{sync} (it is still a blocking call), or we hit \texttt{time}. Now the analysis 
simply processes all the tokens in hand and builds the global type according to the 
same typing rules as before. Figure \ref{graphTwo} shows how the graph structure 
has changed visually from the previous attempt.

To demonstrate further, take the types presented for Snippet \ref{typed}. With this 
algorithm we consume all four tokens presented in the same pass. Because of this we 
can sub-type a \texttt{cue} as needed and the system types correctly. In contrast, 
with the local types for Snippet \ref{untyped}, we can only consume the top token 
from each process on the first iteration. We skip the \texttt{time} marker and are 
then presented with a \texttt{sync} node at the top of each process. Here we can 
successfully report this to be a deadlocking error. Both participants are waiting for a 
message from the other one!

% new graph style

\subsubsection{Global Inference}
As discussed previously in Section \ref{formalST}, the usual approach to multi-party 
session types is to construct the global type of your program and then project the 
individual process types from that. With this one can then write each process, 
ensuring that all communication protocols remain consistent as development proceeds. 
Projection is a simple and decidable approach to session times.

This is not how we handle multi-party session types. In this project we examine 
\emph{Global Inference} as we are building the global type out of a set of pre-written 
local types. This is a difficult problem and is not a subject often approach in this 
field. This report presents a simple, but elegant, way of handling global inference 
in this case. We also present two new constructs, as nothing already available in 
the theory of multi-party session types could describe the situation we required.

Given the dual types \texttt{A:P0P1?} and \texttt{A:P0P1!} we can transform 
this into the global typing \texttt{P0->P1}, representing the fact that a 
message is being transfered from \texttt{P0} to \texttt{P1}. \texttt{A} is not
represented within this global type as it acts as the name of the channel on
which the message is being sent. Standard global types also print the type of 
message being sent; we currently omit this type information. We can consider Sonic 
Pi to be a series of `null' messages being sent on each channel between each 
participant.

The syntax of Sonic Pi allows for some interesting situations,
some of which are not technically typable in standard multi-party session types 
but are perfectly valid, and sometimes useful, situations to occur in Sonic Pi.
\newpage

\paragraph{Replication Type}
The first such situation we consider is when we have several \texttt{cue}
signals with the potential to trigger one \texttt{sync}.

In this scenario the \texttt{sync} in \texttt{P1} may be triggered by either
of the \texttt{cue}s present in the other two threads. Whilst unusual,
it is not unreasonable to suggest this is a valid program. It may be that the
developer has set up the rest of the program code so that threads \texttt{P1} 
and \texttt{P2} alternate for each loop that \texttt{P1} performs. The use
of this in a musical sense could be an interesting drum riff to underpin the
song, or some decorative melodies. To capture this information Sonic Pi 
proposes an `\emph{or}' type, denoted by \texttt{||}. 


\begin{minipage}{\textwidth}
	\begin{lstlisting}[style = sonicpi]
    # P0                # P1                # P2
    in_thread do        in_thread do        in_thread do
      loop do             loop do             loop do
        cue :A              sync :A             cue :A
        sleep 1             sleep 1             sleep 1
      end                 end                 end
    end                 end                 end
	\end{lstlisting}
	\captionof{lstlisting}{One \texttt{sync}, multiple \texttt{cue}}
\end{minipage}


Given this notation the decomposed types\footnote{For the purpose of these discussions
we again omit the \texttt{time} element of the type as it is not relevant.} for this 
set of threads is:
\\
\begin{lstlisting}
  P0 := A:(P0P1)!
  P1 := A:(P0P1 || P2P1)?
  P2 := A:(P2P1)!
\end{lstlisting}

In terms of processing the global type, when each of these types line up
into the same `vertical' relationship, the type constructed has the form 
\texttt{(P0 || P2)->P1}. The algorithm discussed previously does not need
adapting to handle this new type form. The tokens are not removed from the
stored pool until every node in the group has been processed. We are only
required to add in some new code to handle the case where we have detected
the extra information else we may actually print the types as a composition,
\texttt{P0->P1.P2->P1}. This is a valid session type but is not an accurate
definition for this situation.

\paragraph{Broadcasting}
The other situation we considered is when one \texttt{cue} can trigger
multiple \texttt{sync} statements. This is equally as untypable in strict
session types but equally as viable from a musical point of view. The developer
may have multiple melodies playing concurrently that should all sync onto the
same drum beat.

\begin{minipage}{\textwidth}
	\begin{lstlisting}[style = sonicpi]
    # P0                # P1                # P2
    in_thread do        in_thread do        in_thread do
      loop do             loop do             loop do
        sync :A             cue :A              sync :A
        sleep 1             sleep 1             sleep 1
      end                 end                 end
    end                 end                 end
	\end{lstlisting}
	\captionof{lstlisting}{One \texttt{cue}, multiple \texttt{sync}}
\end{minipage}

In this situation we propose a simple `\emph{and}' type. \texttt{cue} is an
asynchronous message that propogates out to all processes currently waiting.
Given this, the \texttt{cue} present in \texttt{P1} will trigger the \texttt{sync}
in \texttt{P0} \emph{and} the \texttt{sync} in \texttt{P2}. The types we 
print for this are as follows:
\\
\begin{lstlisting}
  P0 := A:(P1P0)?
  P1 := A:(P1P0 && P1P2)!
  P2 := A:(P1P2)?
\end{lstlisting}

As before, the algorithm for creating global types from this type set is not
affected. The \texttt{sync} statements will either be in the store from a previous
iteration of the algorithm or be handled as all statements from the same `vertical'
relationship will be processed. The global type for this situation will be printed
as \texttt{P1->(P0 \&\& P2)}. In the event these statements are separated by time
progression, the type will print as the usual global type involving two processes.
\texttt{sync} statements will still be stored as described earlier, either to be
consumed by a later \texttt{cue} of the correct dual typing or to be hit again
and processed as an untypable program.

On initial consideration one might define the \texttt{||}-type as a dual of the
\texttt{\&\&}-type, and vice versa, but this is inaccurate for the context in which
they are used. The main component of these types is still the act of sending or
receiving a message, so dual typings are still based off of the dual relationship
of `\texttt{?}' and `\texttt{!}'. The two types we introduce here are more like
a `typing sugar' to describe situations where a process can send and received
multiple messages at the same time. This is not a situation that session types
usually handle. Session Types are defined by their use of message queues, sending
message packets (be they single packets or multiple) in sequence along the 
\emph{same} channel to \emph{single} receivers. It does not define a situation
where one can trasmit a message on one channel to multiple receivers, better known 
as `broadcasting'. In this case
this projects presents an interesting evolution of session types for this field.
We have implemented multiple modes that our project can operate under to reflect
this, so developers can choose to analysis under standard session typing or under 
the `Sonic Pi' typing outlined here.
\newpage

\subsection{Integration}
The Sonic Pi IDE has two core sections: Server and GUI. The server is written on 
top of the SuperCollider synthesizer and implemented in Ruby whilst the GUI is 
built on the Qt framework and thus written in C++.

Integrating this library comes in two stages. Firstly delving into the server to 
locate the area where the code is evaluated before being made into music. Sonic 
Pi's code evaluation is handled by a \texttt{spider} module, which holds the main 
evaluation method. This module also has the capacity to print straight into the 
IDE's output log widget, making the first step of relaying information to the 
screen very simple. The pseudo code for this \texttt{spider} integretion is very
simple.
\\
\begin{lstlisting}[style = sonicpi]
    def __spider_eval(code)
        # thread setup
        # ...
        code = Preparse(code)
        analysis_output = __verify_code(code)
        __info(analysis_output)
        # ...
        # message handling and thread handling
        # ...
    end
\end{lstlisting}

Secondly, to better display the timing effects information, we choose to define 
a new output widget in the \texttt{mainWindow} class of the gui. Qt, thankfully, 
simplifies the GUI process considerably, meaning it is not too difficult to produce
an attractive addition in keeping with the existing theme of Sonic Pi. We are 
also able to mimic the \texttt{\_\_info} method we use to print to the output log
to be able to print our timing information into our new widget.

% basic outline of sonic pi
% The basic architecture of Sonic Pi can be broken into the server side of the application and the gui. The gui brings a hard dependency on the Qt framework, implemented in C++, whilst the server application is developed with ruby. 
% In order to generate and manipulate sounds, Sonic Pi is implemented in terms f the SuperCollider synthesis server which provides the ability to define arbitiary synthesiers and trigger and manipulate them in real time\cite{AB13}
% The main entry point for Sonic Pi lies in the spider_eval method of the spider class
% defined in the server modules. It is here we intercept the currently played pieces of music and run our own analysis modules.

% once the hookup is done write out the code examples of how messages are passed along
% maybe make a basic diagram of Sonic Pi interactions with/without our project

\newpage

\section{Evaluation}
\thispagestyle{empty}
\subsection{Ruby/C++11 \& Performance}
In evaluating our project it is necessary to question whether the tools used were
in fact the right ones for the given task. We have implemented an analysis library
in C++11 which successfully hooks into the existing Ruby modules. This library
could easily have been written in Ruby, saving the need for the extra steps
of converting the data from one type to another. 

The reason for our
use of C++11 is to take advantage of the speed of the language. The tool used for
for the data transfer here is Ruby Rice, a type-safe and exception-safe interface
between C++ and Ruby's C API\footnote{As found at https://github.com/jasonroelofs/rice}.
Other similar tools for this are Ruby Inline and Ruby's FFI tool. Below is a table
of their relative performances based on some short implementations of fibonacci, 
factorial and pow methods.

\begin{table}[h]
	\centering
	\caption{Ruby Interfacing Performances \cite{amber}} \label{ruby}

	\begin{tabular}{llll}
	          \\
	          & Ruby Inline & Ruby Rice   & Ruby FFI  \\ \hline
	          \\
	factorial & 0.026175138 & 0.197720523 & 0.014882004  \\
	fibonacci & 0.026792521 & 0.202714029 & 0.018928646  \\
	pow       & 0.03252452  & 0.211258897 & 0.023082315
	\end{tabular}
\end{table}

Disappointingly, Ruby Rice performs the slowest of this tools in this scenario. 
On standard desktop environments this amount is little cause for concern as both
the Sonic Pi IDE and the extention library run smoothly with no trouble. This being 
said, it is important to remember that these results are based on very simple 
mathematical methods, not something you would likely use embedded C for in a real 
project. The library is also able to run within the Sonic Pi IDE on its target 
platform with little to no noticeable difference in run time.

While these results are strong it doesn't go the full distance in answering if the
library was better served implemented in Ruby or C++11. The library as it is, is able
to take advantage of many useful features of C++11; notably the host of available
data structures for each task. That said, whilst Ruby only has Array and Hash 
structures avaiable to it, Ruby does still have the concept of classes and modules 
meaning our current class setup is largely portable to a Ruby environment. 
Unfortunately there has not been time thus far to test how fast a Ruby implementation
is compared to the current project state.

Another reason for the use of C++ was the hopes to make greater use of the boost
graph library\footnote{http://www.boost.org/doc/libs/1\_58\_0/libs/graph/doc/}. 
In implementing the graph analysis with this it would have been useful to have
immediate access to such things as the iterators and many different graph algorithms
(djisktra, etc...) that bgl implements. Currently the \texttt{SubGraph} structure
we have does not translate well into bgl. It is possible to compile this 
implementation but data is tricky to initialise and access later on. In the interest
of time this idea was shelved and a much simpler graph structure was put in place.
Despite the large code dependancy bgl can introduce, it may be interesting to
revisit the idea in the future, should the search-algorithms it has ready-implemented
become required.

\subsection{Correctness}
To confirm the correctness of our approach we have built a systematic testing
environment. It is important when building verification tools to have a series
of programs to test results with and ensure any interesting edge cases of behaviour
can be detected and dealt with. In this system we been with a series of simple
programs, testing simple chords and sequencing, single function detection, etc...,
before moving into more complex programs. At the later end of the suite we test
on a full musical piece, taken from the samples provided in the Sonic Pi IDE. In
presenting these results, we provide the source for the program, an expected trace
result and then our actual analysis result.

All codes written here can be repeated with notes in MIDI format (\texttt{:C4}, 
for example) and will produce the same results as the integer format presented.
The reason for this is that the note values are wrapped into \texttt{SymNode} types 
in the same way the numbers are wrapped into the \texttt{IntNode} type. This means 
the analysis approach for them is the same and the results are deterministic.
Session analysis results are only presented when relevant to the code being
discussed, as the library is quite efficient at setting empty types. Where functions
are not being tested, trace information such as whether the statement is a function 
call or not is ommitted for brevity.

\subsubsection{Chords}

\begin{minipage}{\textwidth}
	\begin{lstlisting}[style = sonicpi]
      play 60
      play 62
      play 64
	\end{lstlisting}
	\captionof*{lstlisting}{Chord Test Code}
\end{minipage}
\\
\begin{lstlisting}
    == Expected Trace               == Actual Trace
    [0] -                           [0] -
    [1] conVT: 0, cumVT: 0          [1] conVT: 0, cumVT: 0
    [2] conVT: 0, cumVT: 0          [2] conVT: 0, cumVT: 0
    [3] conVT: 0, cumVT: 0          [3] conVT: 0, cumVT: 0
\end{lstlisting}

This trace correctly confirms that the given program does not advance virtual time 
at any points, reporting a statement length of three.

\subsubsection{Sequences}
\begin{minipage}{\textwidth}
	\begin{lstlisting}[style = sonicpi]
      play 60
      sleep 1
      play 62
      sleep 1
      play 64
	\end{lstlisting}
	\captionof*{lstlisting}{Sequence Test Code}
\end{minipage}
\\
\begin{lstlisting}
    == Expected Trace               == Actual Trace
    [0] -                           [0] -
    [1] conVT: 0, cumVT: 0          [1] conVT: 0, cumVT: 0
    [2] conVT: 1, cumVT: 1          [2] conVT: 1, cumVT: 1
    [3] conVT: 0, cumVT: 1          [3] conVT: 0, cumVT: 1
    [4] conVT: 1, cumVT: 2          [4] conVT: 1, cumVT: 2
    [5] conVT: 0, cumVT: 2          [5] conVT: 0, cumVT: 2
\end{lstlisting}

Here we correctly identify five statements, two of which advance time. The final 
length of virtual time is the sum of all \texttt{sleep} statements used. Here this 
is accurate as it is a simple sequence of operations.

\subsubsection{Loops}
\begin{minipage}{\textwidth}
	\begin{lstlisting}[style = sonicpi]
      loop do
        play 60
        sleep 1
      end
	\end{lstlisting}
	\captionof*{lstlisting}{Loop Test Code}
\end{minipage}
\\
\begin{lstlisting}
    == Expected Trace               == Actual Trace
    [0] -                           [0] -
    [1] conVT: 0, cumVT: 0          [1] conVT: 0, cumVT: 0
    [2] conVT: 0, cumVT: 0          [2] conVT: 0, cumVT: 0
    [3] conVT: 1, cumVT: 1          [3] conVT: 1, cumVT: 1
\end{lstlisting}

The trace identifies three statements, where on is the commencement of the loop. 
By the end of the loop, virtual time will have advanced by one, as printed by 
our analysis.

\begin{minipage}{\textwidth}
	\begin{lstlisting}[style = sonicpi]
      loop do
        play 60
        sleep 1
        loop do
          play 64
          sleep 1
        end
      end
	\end{lstlisting}
	\captionof*{lstlisting}{Nested Loop Test Code}
\end{minipage}
\\
\begin{lstlisting}
    == Expected Trace               == Actual Trace
    [0] -                           [0] -
    [1] conVT:  0, cumVT: 0         [1] conVT:  0, cumVT: 0
    [2] conVT:  0, cumVT: 0         [2] conVT:  0, cumVT: 0
    [3] conVT:  1, cumVT: 1         [3] conVT:  1, cumVT: 1
    [4] conVT:  0, cumVT: 1         [4] conVT:  0, cumVT: 0
    [5] conVT:  0, cumVT: 1         [5] conVT:  0, cumVT: 0
    [6] conVT:  1, cumVT: 2         [6] conVT:  1, cumVT: 1
\end{lstlisting}

This has highlighted a limitation in the current iteration. The contributing time 
detection works but the cumulation does not accurately detect nested blocks. 
Instead it sees each new block and starts the cumulation fresh, as those they were
sequential blocks. This shows the project is not currently making the most use out 
of the block level data that is taken during the construction of the AST. This could 
be fixed by testing whether a new block has a higher level than the previous token.
In the event it does the trace can skip starting a fresh \texttt{cumVT} value, if it 
is equal or less than then the old behaviour applies. 

\begin{minipage}{\textwidth}
	\begin{lstlisting}[style = sonicpi]
      5.times do
        play 60
        sleep 1
      end
	\end{lstlisting}
	\captionof*{lstlisting}{`Timed' Loop Test Code}
\end{minipage}
\\
\begin{lstlisting}
    == Expected Trace               == Actual Trace
    [0] -                           [0] -
    [1] conVT: 5n, cumVT: 0         [1] conVT: -1, cumVT: -1
    [2] conVT:  0, cumVT: 0         [2] conVT:  0, cumVT: -1
    [3] conVT:  1, cumVT: 1         [3] conVT:  1, cumVT:  0
\end{lstlisting}

This loop structure is currently not handled correctly. Ideally, since this is set 
to run a set number of times, once the trace has processed the loop length, there 
should be some stored multiplier that can be used to accurately set the 
\texttt{cumVT} value after the loop to a much higher number (in this test case, the 
next \texttt{cumVT} would be \texttt{5} plus whatever the next \texttt{conVT} 
evaluates to).

What actually happens is \texttt{times} is registered as a function. In ruby terms,
this is technically a function being sent the value 5 to act on. What we may do in 
future is set another keycode (-4) which tells the second pass of the trace that 
there is an argument to this loop that it must multiply the total loop time by and
then use this value in subsequent trace calculations.

\begin{minipage}{\textwidth}
	\begin{lstlisting}[style = sonicpi]
      5.times do
        play 60
        sleep 1
        5.times do
          play 64
          sleep 1
        end
      end
	\end{lstlisting}
	\captionof*{lstlisting}{Nested `Timed' Loop Test Code}
\end{minipage}
\\
\begin{lstlisting}
    == Expected Trace               == Actual Trace
    [0] -                           [0] -
    [1] conVT: 5n, cumVT: 0         [1] conVT: -1, cumVT: -1
    [2] conVT:  0, cumVT: 0         [2] conVT:  0, cumVT: -1
    [3] conVT:  1, cumVT: 1         [3] conVT:  1, cumVT:  0
    [4] conVT: 5n, cumVT: 1         [4] conVT: -1, cumVT: -1
    [5] conVT:  0, cumVT: 1         [5] conVT:  0, cumVT: -1
    [6] conVT:  1, cumVT: 2         [6] conVT:  1, cumVT:  0
\end{lstlisting}

This trace carries forward the limitations of both nested and timed traces shown 
previously.

There is also the ability to make a parameterized loop with \texttt{n.times},
where \texttt{n} has been passed to the function the loop is contained in. This
test is not included here for brevity, as the results are similar those presented
here. The reason for this is whilst the library can see the symbols easily,
the code is currently not processing it. For this reason, parameterized code
is processed as if they were unparameterized. In the case of loops, this means
all loops are handled once and VT is printed as though each loop only occured once.

\subsubsection{Threads}
\begin{minipage}{\textwidth}
	\begin{lstlisting}[style = sonicpi]
      in_thread :foo do
        play 60
        sleep 1
      end
	\end{lstlisting}
	\captionof*{lstlisting}{Thread Test Code}
\end{minipage}
\\
\begin{lstlisting}
    == Expected Trace               == Actual Trace
    [0] -                           [0] -
    [1] conVT: 0, cumVT: 0          [1] conVT: 0, cumVT: 0
    [2] conVT: 0, cumVT: 0          [2] conVT: 0, cumVT: 0
    [3] conVT: 1, cumVT: 1          [3] conVT: 1, cumVT: 1
\end{lstlisting}

Similarly to the simple loop, the analysis presents that this thread has virtual time
one by the end. 

\subsubsection{Data Structures}
\paragraph{Lists}
\begin{minipage}{\textwidth}
	\begin{lstlisting}[style = sonicpi]
      play [60, 62, 64]
	\end{lstlisting}
	\captionof*{lstlisting}{List Test Code}
\end{minipage}

\begin{lstlisting}
    == Expected Trace               == Actual Trace
    [0] conVT: 0, cumVT: 0          [0] conVT: 0, cumVT: 0
\end{lstlisting}

This is a single statement with a different syntax to those usually processed. The 
trace can accurately identify this has no contribution to virtual time despite the 
change in arguments.

\paragraph{Chords \& Scales}
\begin{minipage}{\textwidth}
	\begin{lstlisting}[style = sonicpi]
      chord(:E3, :minor)
	\end{lstlisting}
	\captionof*{lstlisting}{Chord(DS) Test Code}
\end{minipage}

\begin{minipage}{\textwidth}
	\begin{lstlisting}[style = sonicpi]
      scale(:E3, :minor)
	\end{lstlisting}
	\captionof*{lstlisting}{Scale Test Code}
\end{minipage}

Both of these structures have the trace results:
\\
\begin{lstlisting}
    == Expected Trace               == Actual Trace
    [0] conVT: 0, cumVT: 0          [0] conVT: -1, cumVT: 0
\end{lstlisting}

The cumulative times are marked as zero here 
because singular statements are not preceeded by a \texttt{begin} node. This means 
the current information is stored in a \texttt{RootNode}, which by default is 
not processed for information in the current iteration of the project. Running this
again with the statements as part of a statement list give:
\\
\begin{lstlisting}
== First Pass Trace               == Second Pass Trace
[0] conVT: -1, cumVT: (prev-1)    [0] conVT: 0, cumVT: 0
    isCall: true, inFunc: false       isCall: true, inFunc: false
\end{lstlisting}

This shows that \texttt{chord} and \texttt{scale} are currently being detected 
as functions and as they are not defined in our user-function list, they are being 
filled as a zero time statements. Whilst they technically are zero sleep function 
calls, it raises the question as to whether we should only mark user defined 
functions or if it is still useful to mark the statement. 

\paragraph{Rings}
\begin{minipage}{\textwidth}
	\begin{lstlisting}[style = sonicpi]
      (ring 52, 55, 59)
	\end{lstlisting}
	\captionof*{lstlisting}{Ring V1 Test Code}
\end{minipage}
\\
\begin{lstlisting}
    == Expected Trace               == Actual Trace
    [0] conVT: 0, cumVT: 0          [0] conVT: 0, cumVT: 0
    [1] conVT: 0, cumVT: 0          [1] conVT: 0, cumVT: 0
\end{lstlisting}

This function call is in a similar state to the list test run previously. The 
parentheses cause the AST in this case to present a root \texttt{begin} node in 
every case, rather than allowing a single statement as before. Again, the analysis 
is able to identify this has no effect on the current virtual time and is therefore 
correct. 

\begin{minipage}{\textwidth}
	\begin{lstlisting}[style = sonicpi]
      [52, 55, 59].ring
	\end{lstlisting}
	\captionof*{lstlisting}{Ring V2 Test Code}
\end{minipage}

The results for this style of function call are the same as those just discussed 
for chords and scales. Other data structure functions (\texttt{range}, 
\texttt{bools}, \texttt{knit}, and \texttt{spread}) have similar results to 
these.

\subsubsection{Dead Code}
\begin{minipage}{\textwidth}
	\begin{lstlisting}[style = sonicpi]
      loop do
        play 60
        sleep 1
      end

      loop do
        play 60
        sleep 1
      end
	\end{lstlisting}
	\captionof*{lstlisting}{Dead Code V1 Test Code}
\end{minipage}
\\
\begin{lstlisting}
    == Expected Trace               == Actual Trace
    [0] -                           [0] -
    [1] conVT:  0, cumVT: 0         [1] conVT:  0, cumVT: 0
    [2] conVT:  0, cumVT: 0         [2] conVT:  0, cumVT: 0
    [3] conVT:  1, cumVT: 1         [3] conVT:  1, cumVT: 1
    [4] conVT: -3, cumVT: 0         [4] conVT: -3, cumVT: 0
    [5] conVT: -3, cumVT: 0         [5] conVT: -3, cumVT: 0
    [6] conVT: -3, cumVT: 1         [6] conVT: -3, cumVT: 1
\end{lstlisting}

\texttt{-3} is the keycode used to show deadcode. The output in the IDE can use 
this to print blank statements rather than numbers, helping the developer see
when some code is inaccessable. This trace also shows sequential cumulation is 
being calculated accurately as the trace does not waste time fixing the 
\texttt{cumVT} values processed during the first pass.

\begin{minipage}{\textwidth}
	\begin{lstlisting}[style = sonicpi]
      loop do
        play 60
        sleep 1
        loop do
          play 64
          sleep 1
        end
        play 66
        sleep 1
      end
	\end{lstlisting}
	\captionof*{lstlisting}{Dead Code V2 Test Code}
\end{minipage}
\\
\begin{lstlisting}
    == Expected Trace               == Actual Trace
    [0] -                           [0] -
    [1] conVT:  0, cumVT: 0         [1] conVT:  0, cumVT: 0
    [2] conVT:  0, cumVT: 0         [2] conVT:  0, cumVT: 0
    [3] conVT:  1, cumVT: 1         [3] conVT:  1, cumVT: 1
    [4] conVT:  0, cumVT: 1         [4] conVT:  0, cumVT: 0
    [5] conVT:  0, cumVT: 1         [5] conVT:  0, cumVT: 0
    [6] conVT:  1, cumVT: 2         [6] conVT:  1, cumVT: 1
    [7] conVT: -3, cumVT: 2         [7] conVT: -3, cumVT: 1
    [8] conVT: -3, cumVT: 3         [8] conVT: -3, cumVT: 2
\end{lstlisting}

This also highlights the error found earlier in collecting time for nested loops, 
but the dead code is still detected properly.

The type of dead code that the trace cannot comment on is when a conditional is 
never going to enter a certain branch. This analysis does not ever evaluate the 
results of conditionals, so it could be that we mark the \texttt{ifNode} with the 
virtual time of a branch it will never reach. The developer will have to notice this 
themselves during the performance when their music always runs for a length of 
virtual time shorter than what the analysis reports.

\subsubsection{Function Calls} \label{evalFunc}
\paragraph{Function Definition}
\begin{minipage}{\textwidth}
	\begin{lstlisting}[style = sonicpi]
      define :func do
        play 55
        sleep 1
      end
	\end{lstlisting}
	\captionof*{lstlisting}{Function Definition Test Code}
\end{minipage}
\\
\begin{lstlisting}
== Expected Trace                 == Actual Trace
[0] -                             [0] -
[1] conVT:  0, cumVT: 0,          [1] conVT:  0, cumVT: 0
    isCall: false, inFunc: true       isCall: false, inFunc: true            
[2] conVT:  0, cumVT: 0           [2] conVT:  0, cumVT: 0
    isCall: false, inFunc: true       isCall: false, inFunc: true 
[3] conVT:  1, cumVT: 1           [3] conVT:  1, cumVT: 1
    isCall: false, inFunc: true       isCall: false, inFunc: true 

Function: func 1                  Function: func 1
\end{lstlisting}

The analysis will print out a list of those function definiitions found during 
the analysis. Here we have correctly identified that a function, called 
\texttt{func} was defined at statement index 1. We can also see it has correctly 
identified which statements where inside this function and the virtual time data 
throughout is accurate.

\paragraph{Function Detection}
\begin{minipage}{\textwidth}
	\begin{lstlisting}[style = sonicpi]
      define :func do
        play 55
        sleep 1
      end

      func
	\end{lstlisting}
	\captionof*{lstlisting}{Function Detection Test Code}
\end{minipage}
\\
\begin{lstlisting}
== Expected Trace                 == Actual Trace
[0] -                             [0] -
[1] conVT:  0, cumVT: 0,          [1] conVT:  0, cumVT: 0
    isCall: false, inFunc: true       isCall: false, inFunc: true            
[2] conVT:  0, cumVT: 0           [2] conVT:  0, cumVT: 0
    isCall: false, inFunc: true       isCall: false, inFunc: true 
[3] conVT:  1, cumVT: 1           [3] conVT:  1, cumVT: 1
    isCall: false, inFunc: true       isCall: false, inFunc: true 
[4] conVT:  1, cumVT: 1           [4] conVT:  1, cumVT: 1
    isCall: true, inFunc: false       isCall: true, inFunc: false 

Function: func 1                  Function: func 1
\end{lstlisting}

Here we extend the previous test to acctually call the function defined. Index 
4 is identified as the statement with a function call and the analysis correctly 
marks this as outside of the block of the previous function. The first pass of 
the trace was as follows:
\\
\begin{lstlisting}
    == First Pass Trace
    [0] -
    [1] conVT:  0, cumVT: 0, isCall: false, inFunc:  true
    [2] conVT:  0, cumVT: 0, isCall: false, inFunc:  true
    [3] conVT:  1, cumVT: 1, isCall: false, inFunc:  true
    [4] conVT: -1, cumVT: 0, isCall:  true, inFunc: false
\end{lstlisting}

This is provided to show the analysis correctly marked index 4 as being a function 
call and reset the \texttt{cumVT} as it had left a function block from the previous 
line.

\begin{minipage}{\textwidth}
	\begin{lstlisting}[style = sonicpi]
      define :foo do      
        play 55
        sleep 1
      end

      play 60
      foo
      bar

      define :bar do
        play 75
        sleep 2
      end
	\end{lstlisting}
	\captionof*{lstlisting}{Multiple Function Detection Test Code}
\end{minipage}
\\
\begin{lstlisting}
== Expected Trace                 == Actual Trace
[0] -                             [0] -
[1] conVT: 0, cumVT: 0,           [1] conVT: 0, cumVT: 0
    isCall: false, inFunc:  true      isCall: false, inFunc:  true            
[2] conVT: 0, cumVT: 0            [2] conVT: 0, cumVT: 0
    isCall: false, inFunc:  true      isCall: false, inFunc:  true 
[3] conVT: 1, cumVT: 1            [3] conVT: 1, cumVT: 1
    isCall: false, inFunc:  true      isCall: false, inFunc:  true 
[4] conVT: 0, cumVT: 0            [4] conVT: 0, cumVT: 0
    isCall: false, inFunc: false      isCall: false, inFunc: false 
[5] conVT: 1, cumVT: 1            [5] conVT: 1, cumVT: 1
    isCall:  true, inFunc: false      isCall:  true, inFunc: false 
[6] conVT: 2, cumVT: 3            [6] conVT: 2, cumVT: 3
    isCall:  true, inFunc: false      isCall:  true, inFunc: false 
[7] conVT: 0, cumVT: 0            [7] conVT: 0, cumVT: 0
    isCall: false, inFunc:  true      isCall: false, inFunc:  true 
[8] conVT: 0, cumVT: 0            [8] conVT: 0, cumVT: 0
    isCall: false, inFunc:  true      isCall: false, inFunc:  true 
[9] conVT: 2, cumVT: 2            [9] conVT: 2, cumVT: 2
    isCall: false, inFunc:  true      isCall: false, inFunc:  true 

Function: foo 1                   Function: foo 1
          bar 7                             bar 7
\end{lstlisting}

This demonstrate the accuracy of the function detection, as 
the trace is able to correctly fill the virtual times whether the function is 
defined before or after its uses.
\\
\begin{lstlisting}
    == First Pass Trace (Differences)
    [5] conVT: -1, cumVT: -1, isCall: true, inFunc: false 
    [6] conVT: -1, cumVT: -2, isCall: true, inFunc: false 
\end{lstlisting}

\paragraph{Nested Functions}
\begin{minipage}{\textwidth}
	\begin{lstlisting}[style = sonicpi]
      define :bottom do
        sleep 10
      end
    
      define :top do      
        sleep 2
        bottom
      end

      top
      bottom
	\end{lstlisting}
	\captionof*{lstlisting}{Multiple Function Detection Test Code}
\end{minipage}
\\
\begin{lstlisting}
== Expected Trace                 == Actual Trace
[0] -                             [0] -
[1] conVT:  0, cumVT:  0,         [1] conVT:  0, cumVT:  0
    isCall: false, inFunc:  true      isCall: false, inFunc:  true            
[2] conVT: 10, cumVT: 10          [2] conVT: 10, cumVT: 10
    isCall: false, inFunc:  true      isCall: false, inFunc:  true 
[3] conVT:  0, cumVT:  0          [3] conVT:  0, cumVT:  0
    isCall: false, inFunc:  true      isCall: false, inFunc:  true 
[4] conVT:  2, cumVT:  2          [4] conVT:  2, cumVT:  2
    isCall: false, inFunc:  true      isCall: false, inFunc:  true 
[5] conVT: 10, cumVT: 12          [5] conVT: 10, cumVT: 12
    isCall:  true, inFunc:  true      isCall:  true, inFunc:  true 
[6] conVT: 12, cumVT: 12          [6] conVT: 12, cumVT: 12
    isCall:  true, inFunc: false      isCall:  true, inFunc: false 
[7] conVT: 10, cumVT: 22          [7] conVT: 10, cumVT: 22
    isCall:  true, inFunc: false      isCall:  true, inFunc: false 

Function: bottom 1                   Function: top 1
          top 3                          bottom 3
\end{lstlisting}

This demonstrates that both sequential functions and nested functions are handled 
as required. Given how Sonic Pi's scoping operates, functions will always be defined 
before they can be called, meaning a nested function can always be given a time when 
calculating the time of the upper function.

\paragraph{Parameterized Functions}
\begin{minipage}{\textwidth}
	\begin{lstlisting}[style = sonicpi]
      define :func do |n|
        play 60
        sleep n
      end
	\end{lstlisting}
	\captionof*{lstlisting}{Parametre Test Code}
\end{minipage}

This feature is not implemented correctly. To explain the flaws in the current 
iteration we will present only the first pass information and explain why the second 
pass cannot produce the correct result.
\\
\begin{lstlisting}
    == First Pass Trace
    [1] conVT:  0, cumVT:  0, isCall: false, inFunc: false
    [2] conVT: -1, cumVT: -1, isCall: false, inFunc:  true
    [3] conVT:  0, cumVT: -1, isCall: false, inFunc:  true
    [4] conVT: -1, cumVT: -2, isCall: false, inFunc:  true
\end{lstlisting}

In this syntax the first line is processed as having a function call (for 
the keyword \texttt{do}), keycode \texttt{-1}. This identifies a problem with the 
way the analysis marks statements. \texttt{-1} is overloaded as the default symbol 
recognition and a marker as a function call. To be clear when something is 
unrecognised the default keycode should be changed. In the second pass, index 2 
is zero'd and index 4 remains the same. This is due to the fact that some code 
exists which resets the values for the first method call found. This is useful 
code for sequences of method calls where the incorrect starting times would 
sometimes garble the results, but this test highlights two things: it should not 
actually be run if the function in question doesn't exist (so that the line 
remains unchanged for clarity after the fact), and it does not reset properly 
between sequences of functions.

In the future the analysis will need to recognise when the \texttt{block} node's
\texttt{arg} child has children of it's own. This marks when something has been 
written with parametres and what those parametre symbols are. They can then be 
stored in the scope of that block similar to how functions are stored at the global 
level of the trace. 

\subsubsection{Conditionals} \label{evalCond}
\begin{minipage}{\textwidth}
	\begin{lstlisting}[style = sonicpi]
      if cond then
        sleep 1
      else
        sleep 0.5
      end
	\end{lstlisting}
	\captionof*{lstlisting}{Conditional Test Code}
\end{minipage}
\\
\begin{lstlisting}
    == Expected Trace               == Actual Trace
    [0] conVT:  1, cumVT:    1      [0] conVT:  1, cumVT:    1
    [1] conVT: -3, cumVT:   -1      [1] conVT: -3, cumVT:   -1
    [2] conVT: -3, cumVT: -1.5      [2] conVT: -3, cumVT: -1.5
\end{lstlisting}

Conditionals presented an interesting case while collecting these test results. The
\texttt{if} keyword actually works similar to the \texttt{block} keyword in that 
they both mark the beginning of some statement list. This means they will always 
come before a \texttt{begin} keyword in the tree. For the most part during a program 
this does not cause much issue as the processing is still the same. However, in 
this case, where we have only the conditional in our workspace, the \texttt{if} gets 
taken in as the \texttt{RootNode} in our AST. Previously \texttt{RootNode} was 
not involved in the processing as it was assumed there would never be anything but 
a \texttt{begin} or a \texttt{block} keyword there. Because of this, the second 
pass could not accurately clean the trace and our results look like this:
\\
\begin{lstlisting}
    == Trace
    [0] conVT:  -2, cumVT:   -2
    [1] conVT:   1, cumVT:   -1
    [2] conVT: 0.5, cumVT: -0.5
\end{lstlisting}

This has since been fixed, hence the corrected trace shown initially.

Given the nature of our implementation (as described in Section 4), this has the
same output regardless of the conditional's expression.

\subsubsection{\texttt{;} Separation}
\begin{minipage}{\textwidth}
	\begin{lstlisting}[style = sonicpi]
      play 60 ; sleep 1 ; play 66 ; sleep 0.5
	\end{lstlisting}
	\captionof{lstlisting}{Multiple Statements on Single Line Test Code}
\end{minipage}
\\
\begin{lstlisting}
    == Expected Trace               == Actual Trace
    [0] -                           [0] -
    [1] conVT:   0, cumVT:   0      [1] conVT:   0, cumVT:   0
    [2] conVT:   1, cumVT:   1      [2] conVT:   1, cumVT:   1
    [3] conVT:   0, cumVT:   1      [3] conVT:   0, cumVT:   1
    [4] conVT: 0.5, cumVT: 1.5      [4] conVT: 0.5, cumVT: 1.5
\end{lstlisting}

To further demonstrate the ability to process multiple statements on one line,
and that the analysis will work whether statements are comma separated or not. The
analysis recognises floats and successfully as ints and time advancement is summed
correctly throughout the whole trace.

\subsubsection{Communications}
\begin{minipage}{\textwidth}
	\begin{lstlisting}[style = sonicpi]
      live_loop :foo do
        cue :A
        sleep 1
      end
	\end{lstlisting}
	\captionof{lstlisting}{Cue Detection Test Code}
\end{minipage}
\\
\begin{lstlisting}
    == Expected Trace               == Actual Trace
    [0] -                           [0] -
    [1] conVT: 0, cumVT: 0          [1] conVT: 0, cumVT: 0
    [2] conVT: 0, cumVT: 0          [2] conVT: 0, cumVT: 0
    [3] conVT: 1, cumVT: 1          [3] conVT: 1, cumVT: 1

    == Expected Arc Results          == Actual Arc Results 
    Cue 0 Block:P0 foo               Cue 0 Block:P0 foo
      |_> points at                    |_> points at 
    Cue 1 Block:P0 A                 Cue 1 Block:P0 A
      |_> points at                    |_> points at 
    
    '' Unclaimed Syncs ''            '' Unclaimed Syncs '' 
\end{lstlisting}

\texttt{cue} is accurately recorded in the session graph whilst virtual time is 
also still recorded correctly. \texttt{live\_loop}s also send out a \texttt{cue} 
operation and this is also present.

\begin{minipage}{\textwidth}
	\begin{lstlisting}[style = sonicpi]
      live_loop :foo do
        sync :A
        sleep 1
      end
	\end{lstlisting}
	\captionof{lstlisting}{Sync Detection Test Code}
\end{minipage}
\\
\begin{lstlisting}
    == Expected Trace               == Actual Trace
    [0] -                           [0] -
    [1] conVT: 0, cumVT: 0          [1] conVT: 0, cumVT: 0
    [2] conVT: 0, cumVT: 0          [2] conVT: 0, cumVT: 0
    [3] conVT: 1, cumVT: 1          [3] conVT: 1, cumVT: 1

    == Expected Arc Results          == Actual Arc Results 
    Cue 0 Block:P0 foo               Cue 0 Block:P0 foo
      |_> points at                    |_> points at 
      
    '' Unclaimed Syncs ''            '' Unclaimed Syncs '' 
    = 0 A                            = 0 A
\end{lstlisting}

\texttt{sync} is recorded properly alongside the correct virtual time results. 
\texttt{sync} is recorded into a different data structure when not pared with a 
\texttt{cue} as it is blocking and will disrupt your performance. The separate 
data structure is easier to print as an error result if desired.

\begin{minipage}{\textwidth}
	\begin{lstlisting}[style = sonicpi]
      in_thread do         in_thread do
        loop do              loop do
          sync :A              cue :A 
          cue :B               sync :B
          sleep 1              play 60
          play 63              sleep 0.5
        end                  end
      end                  end
	\end{lstlisting}
	\captionof{lstlisting}{Global Type Test Code}
\end{minipage}
\\
\begin{lstlisting}
   == Expected Trace               == Actual Trace
   [0] -                           [0] -
   [1]  conVT:   0, cumVT:   0     [1]  conVT:   0, cumVT:   0
   [2]  conVT:   0, cumVT:   0     [2]  conVT:   0, cumVT:   0
   [3]  conVT:   0, cumVT:   0     [3]  conVT:   0, cumVT:   0
   [4]  conVT:   0, cumVT:   0     [4]  conVT:   0, cumVT:   0
   [5]  conVT:   1, cumVT:   1     [5]  conVT:   1, cumVT:   1
   [6]  conVT:   0, cumVT:   1     [6]  conVT:   0, cumVT:   1
   [7]  conVT:   0, cumVT:   0     [7]  conVT:   0, cumVT:   0
   [8]  conVT:   0, cumVT:   0     [8]  conVT:   0, cumVT:   0
   [9]  conVT:   0, cumVT:   0     [9]  conVT:   0, cumVT:   0
   [10] conVT:   0, cumVT:   0     [10] conVT:   0, cumVT:   0
   [11] conVT:   0, cumVT:   0     [11] conVT:   0, cumVT:   0
   [12] conVT: 0.5, cumVT: 0.5     [12] conVT: 0.5, cumVT: 0.5

    == Expected Arc Results          == Actual Arc Results 
    Cue 1 Block:P0 B                 Cue 1 Block:P0 B
       |_> points at                    |_> points at 
           -> P1: 4 B                       -> P1: 4 B
    Cue 3 Block:P1 A                 Cue 3 Block:P1 A
       |_> points at                    |_> points at 
           -> P0: 0 A                       -> P0: 0 A

    '' Unclaimed Syncs ''            '' Unclaimed Syncs '' 

    == Expected Local Types          == Actual Local Types
    SubGraph P0                      SubGraph P0
      A:(P1P0)?.B:(P0P1)!.time         A:(P1P0)?.B:(P0P1)!.time
    SubGraph P1                      SubGraph P1
      A:(P1P0)!.B:(P0P1)?.time         A:(P1P0)!.B:(P0P1)?.time

    Global Type: P1->P0.P0->P1       Global Type: P1->P0.P0->P1
\end{lstlisting}

This trace shows the virtual time kept during the program, the \texttt{cue} operations 
by index and which \texttt{sync} operations they link with. The trace accurately 
reports no unclaimed \texttt{sync}s. After this are the local types of each thread, 
as named by our analysis, and the global type of the system. All traces shows that 
the analysis information is correct; no results disagree with one another.

\begin{minipage}{\textwidth}
	\begin{lstlisting}[style = sonicpi]
      in_thread do         in_thread do
        loop do              loop do
          sync :A              sync :B 
          cue :B               cue :A 
          sleep 1              play 60
          play 63              sleep 0.5
        end                  end
      end                  end
	\end{lstlisting}
	\captionof{lstlisting}{Deadlock Detection Test Code}
\end{minipage}
\\
\begin{lstlisting}
   == Expected Trace               == Actual Trace
   [0] -                           [0] -
   [1]  conVT:   0, cumVT:   0     [1]  conVT:   0, cumVT:   0
   [2]  conVT:   0, cumVT:   0     [2]  conVT:   0, cumVT:   0
   [3]  conVT:   0, cumVT:   0     [3]  conVT:   0, cumVT:   0
   [4]  conVT:   0, cumVT:   0     [4]  conVT:   0, cumVT:   0
   [5]  conVT:   1, cumVT:   1     [5]  conVT:   1, cumVT:   1
   [6]  conVT:   0, cumVT:   1     [6]  conVT:   0, cumVT:   1
   [7]  conVT:   0, cumVT:   0     [7]  conVT:   0, cumVT:   0
   [8]  conVT:   0, cumVT:   0     [8]  conVT:   0, cumVT:   0
   [9]  conVT:   0, cumVT:   0     [9]  conVT:   0, cumVT:   0
   [10] conVT:   0, cumVT:   0     [10] conVT:   0, cumVT:   0
   [11] conVT:   0, cumVT:   0     [11] conVT:   0, cumVT:   0
   [12] conVT: 0.5, cumVT: 0.5     [12] conVT: 0.5, cumVT: 0.5

    == Expected Arc Results          == Actual Arc Results 
    Cue 1 Block:P0 B                 Cue 1 Block:P0 B
       |_> points at                    |_> points at 
           -> P1: 3 B                       -> P1: 3 B
    Cue 3 Block:P1 A                 Cue 3 Block:P1 A
       |_> points at                    |_> points at 
           -> P0: 0 A                       -> P0: 0 A

    '' Unclaimed Syncs ''            '' Unclaimed Syncs '' 

    == Expected Local Types          == Actual Local Types
    SubGraph P0                      SubGraph P0
      A:(P1P0)?.B:(P0P1)!.time         A:(P1P0)?.B:(P0P1)!.time
    SubGraph P1                      SubGraph P1
      B:(P0P1)?.A:(P1P0)!.time         B:(P0P1)?.A:(P1P0)!.time

    Global Type: -                   Global Type: -
\end{lstlisting}

Similar to previously, this trace shows that the analysis can correctly identify 
a deadlock situation as the global type of the system as not been printed. 

\subsubsection{Musical Score}
\begin{minipage}{\textwidth}
	\begin{lstlisting}[style = sonicpi]
 load_samples [:drum_heavy_kick, :drum_snare_soft]

 define :drums do
   cue :slow_drums
   6.times do
     sample :drum_heavy_kick, rate: 0.8
     sleep 0.5
   end
   cue :fast_drums
   8.times do
     sample :drum_heavy_kick, rate: 0.8
     sleep 0.125
   end
 end
 
 define :snare do
   cue :snare
   sample :drum_snare_soft
   sleep 1
 end
 
 define :synths do
   puts "how does it feel?"
   use_synth :mod_saw
   use_synth_defaults amp: 0.5, attack: 0, sustain: 1, 
                      release: 0.25, cutoff: 90, mod_range: 12, 
                      mod_phase: 0.5, mod_invert_wave: 1
   notes = [:F, :C, :D, :D, :G, :C, :D, :D]
   notes.each do |n|
     play note(n, octave: 1)
     play note(n, octave: 2)
     sleep 1
   end
 end
    \end{lstlisting}
\end{minipage}

\begin{minipage}{\textwidth}
	\begin{lstlisting}[style = sonicpi]
      in_thread(name: :synths) do
        sleep 6
        loop{synths}
      end
      
      in_thread(name: :drums) do
        loop{drums}
      end
      
      in_thread(name: :snare) do
        sleep 12.5
        loop{snare}
      end
	\end{lstlisting}
	\captionof*{lstlisting}{Monday Blues - Coded by Sam Aaron, as found in Sonic 
	Pi IDE Samples}
\end{minipage}

This test result is lengthy but is included here in full, with full source, as it 
is the best example of the current state of this project. We do not present the second 
pass trace in full; instead we inline it with the relevant discussion points. 
This is an accurate example of some piece that may be written for a real live coding 
performance and contains both complicated nesting structures and multiple function 
definitions. It does not make full use of the communication primitives, using only 
the \texttt{cue} operation in this test. This seems illogical, as it has nothing to 
work with, but \texttt{cue} does print to the output box of the Sonic Pi IDE, making 
it another useful performance resource, and a situation the analysis should be able 
to handle.
\\
\begin{lstlisting}
  == First Pass Trace
  [0] -
  [1]  conVT:    -1, cumVT:     -1, isCall:  true, inFunc: false
  [2]  conVT:    -1, cumVT:     -2, isCall:  true, inFunc: false
  [3]  conVT:     0, cumVT:      0, isCall: false, inFunc:  true
  [4]  conVT:     0, cumVT:      0, isCall: false, inFunc:  true
  [5]  conVT:    -1, cumVT:     -1, isCall:  true, inFunc:  true
  [6]  conVT:     0, cumVT:     -1, isCall: false, inFunc: false
  [7]  conVT:   0.5, cumVT:   -1.5, isCall: false, inFunc:  true
  [8]  conVT:     0, cumVT:   -1.5, isCall: false, inFunc: false
  [9]  conVT:     0, cumVT:   -1.5, isCall:  true, inFunc: false
  [10] conVT:     0, cumVT:   -1.5, isCall:  true, inFunc: false
  [11] conVT: 0.125, cumVT: -1.375, isCall: false, inFunc: false
  [12] conVT:     0, cumVT:      0, isCall: false, inFunc:  true
  [13] conVT:     0, cumVT:      0, isCall: false, inFunc:  true
  [14] conVT:     0, cumVT:      0, isCall: false, inFunc:  true
  [15] conVT:     1, cumVT:      1, isCall: false, inFunc: false
  [16] conVT:     0, cumVT:      0, isCall: false, inFunc:  true
  [17] conVT:    -1, cumVT:     -1, isCall:  true, inFunc:  true
  [18] conVT:    -1, cumVT:     -2, isCall:  true, inFunc:  true
  [19] conVT:    -1, cumVT:     -3, isCall:  true, inFunc:  true
  [20] conVT:    -1, cumVT:     -4, isCall:  true, inFunc:  true
  [21] conVT:     0, cumVT:      0, isCall: false, inFunc:  true
  [22] conVT:     0, cumVT:      0, isCall:  true, inFunc:  true
  [23] conVT:     0, cumVT:      0, isCall: false, inFunc:  true
  [24] conVT:     0, cumVT:      0, isCall:  true, inFunc:  true
  [25] conVT:     1, cumVT:      1, isCall: false, inFunc:  true
  [26] conVT:    -1, cumVT:      0, isCall:  true, inFunc:  true
  [27] conVT:     6, cumVT:     -6, isCall: false, inFunc:  true
  [28] conVT:     0, cumVT:      0, isCall: false, inFunc:  true
  [29] conVT:    -1, cumVT:     -1, isCall:  true, inFunc:  true
  [30] conVT:    -1, cumVT:     -2, isCall:  true, inFunc:  true
  [31] conVT:     0, cumVT:      0, isCall: false, inFunc:  true
  [32] conVT:    -1, cumVT:     -1, isCall:  true, inFunc:  true
  [33] conVT:    -1, cumVT:     -2, isCall:  true, inFunc:  true
  [34] conVT:  12.5, cumVT:   10.5, isCall: false, inFunc:  true
  [35] conVT:     0, cumVT:      0, isCall: false, inFunc:  true
  [36] conVT:    -1, cumVT:     -1, isCall:  true, inFunc:  true

  Function: drums 3
            snare 12
            synths 16

    == Arc Results       
    Cue 0 Block:P0 slow_drums
       |_> points at 
    Cue 1 Block:P1 fast_drums
       |_> points at 
    Cue 3 Block:P2 snare
       |_> points at 

    '' Unclaimed Syncs ''   

    == Local Types
    SubGraph P0
      slow_drums:()!
    SubGraph P1
      time.fast_drums:()!
    SubGraph P2
      time.snare:()!.time
    SubGraph P3
      time


    Global Type: -  
\end{lstlisting}

The main failing with this piece is the number of syntax structures that were not 
previously considered when first working with the program. The analysis marks many
statements as functions (\texttt{puts}, \texttt{sample}, \texttt{use\_synth}) but 
has no record of them in the function definitions list. Given this in the second pass 
it updates most of these records with the incorrect virtual time contributions and 
can't calculate the accumulated amounts correctly. The analysis does detect each 
individual use of \texttt{sleep} correctly but total running time of this program 
would not be possible until the issues involing parameterized loops and functions 
were solved.

In communication terms, the analysis can detect the use of \texttt{cue} and 
\texttt{sync} but has highlighted a problem with their use in functions. The graph is 
not splitting across the blocks correctly, suggesting nested loops do not build 
correctly, and the graph does not distinguish between block types. This highlights 
that the session type analysis should also build a function list in the same way 
as the timing effects system did. Then, when the function is used in a process the 
nodes can be assigned to the correct process \texttt{SubGraph}.

\subsection{Visual Adaptation}
% screenshot here
\begin{figure}[ht]
	\centering
	\includegraphics[width=\textwidth]{images/sonic-ide-v2.png}
	\caption{Sonic Pi V2.1.1 IDE Extension - Apple Mac View} \label{macview}
\end{figure}

The visual differences to the IDE are quite minimal. One notable feature that is 
currently missing is the ability not to load this library whilst using the IDE, a 
useful switch to have as not all users will want to see this information all of the
time. 

For the most part test users relayed that they found the virtual time bar to the 
left to be a handy little tool. Some noted that the printed session types in the 
output log were well placed but did not have enough experience with the material 
to be able to get the most use out of the information. 

One interesting piece of 
feedback was the lack of ability to see the virtual time of your program before 
running it for the first time. At the moment most of the analysis only runs 
when playing the piece, so there is no trace to display until that point. It may 
be that in the future the integration with the library is improved such that, 
similarly to how a \texttt{live\_loop} can update the synth mid-performance,
the IDE can detect a change in the workspace and run the analysis on the new source. 
This would require a couple of improvements to the analysis itself; currently the 
analysis runs afresh each time the code is processed. In the future it would be
preferable to hold a trace object for each workspace and to dynamically update 
this during the lifetime of the IDE window. 

% some self evaluation here?

% ease of use when put into sonic-pi
% use of information
% technical and non technical

\newpage

\section{Conclusion}
\thispagestyle{empty}
In this chapter we bring the project to a close by first discussing various 
improvements for the project, and other features that we might implement 
in the future, before concluding with a summary of achievements. 

\subsection{Future Work}
\paragraph{Function Parametres}
In its current state, the project is able to detect the use of functions
and accurately update the current program state to show the current virtual
time at a given point. That said, it is not able to do so with variable
amounts of sleep. When a function is called with an argument that affects the
amount of time the process must sleep for, the project cannot detect the
argument passed into the function and apply this where necessary.

\paragraph{Function Calls for Sessions}
As noted in Section \ref{formalST}, the Session Type analysis of this project is 
unable to handle function calls within processes. Currently this limitation extends 
to the fact the analysis will consider function definitions as their own processes. 
In the future this project will benefit from employing similar techniques when 
building the graph structure as when building the timing trace. One option to explore 
is the idea that the graph collect lists of nodes as defined in the functions and 
insert these into the process graph when the functions are called. 

\paragraph{Branching Session Types}
This iteration of the project cannot accurately handle branching statements 
within a program. There is some interesting theory to be considered in this
improvement as labels will not be passed between processes as standard Session
Types may expect them to be. Instead it is more likely that a given if-statement
will be considered as an unlabelled branching type and its condition
can be marked as the dual selection type repesenting all possible label choices.
The subsequent global type for this statement appears as an internal message
from the process to itself; this is not an accurate application of branching
types but suits the given situation neatly.

\paragraph{Detailed Time Nodes}
In the current iteration of the session graph, the passage of time is simply
denoted by an empty \texttt{GraphNode} with no extra information. In most cases
this works well but in the event that there are two processes, \texttt{P0} and
\texttt{P1}, it may be that \texttt{P0} processes two passages of time of equal
length two one passage in \texttt{P1}. In this case the current node of 
\texttt{P1} will be testing itself against the wrong position in \texttt{P0} 
leading to an inaccurate type analysis. In the future we could investigate
incorporating more of the information calculated in the timing affects portion
of the program into the construction of the session graph.

\paragraph{Minecraft API}
Since starting this project, the Sonic Pi IDE has added a section detailing how 
to use Sonic Pi alongside the Minecraft API. Our analysis is currently untested 
against this API and it may prove interesting in the future to adapt the library 
to be able to run alongside this API. 

\subsection{Achievement Summary}
We have produced a lightweight shared library that can statically analysis a 
Sonic Pi program. We have made a tool that can assist the developer with both temporal 
and concurrent reasoning of their Sonic Pi programs. The tool will output information 
on both the state of virtual time throughout the source, and also output information 
about the communication structure of the program. It can successfully report when 
the program is deadlocked. We also contribute two new typing ideas for multi-party 
session types with our \texttt{\&\&} and \texttt{||} types, which handle replication 
type and broadcasting situations within Sonic Pi. 

This library can integrate successfully with the existing Sonic Pi IDE and we 
have also provided a small systematic testing environment within the project 
to prove the correctness of our analysis and demonstrate the current limitations 
of the project. With the current set of features implemented, the library is able 
to accurately run during live perforamnces with very simple musical pieces. Working 
with fuller pieces, like those presented in the Evaluation, is also very close to 
being a reality.
\newpage


\begin{thebibliography}{9}
\thispagestyle{empty}

\bibitem{naec}
  \emph{http://www.naec.org.uk/events}

\bibitem{sp}
  \emph{http://sonic-pi.net/}

\bibitem{rp}
  \emph{http://www.raspberrypi.org/about/}

\bibitem{micro}
  \emph{http://news.microsoft.com/apac/2015/03/23/three-out-of-four-students-in-asia-pacific-want-coding-as-a-core-subject-in-school-reveals-microsoft-study/}
  % (Visited on 8/6/2015)

\bibitem{crush}
  \emph{http://www.curse.com/mc-mods/minecraft?filter-project-game-version=} \\
  (Visited on 14/6/2015)

\bibitem{amber}
  \emph{https://www.amberbit.com/blog/2014/6/12/calling-c-cpp-from-ruby/}

\bibitem{AB13}
  Aaron, S., and Blackwell, A.F.,
  \emph{From Sonic Pi to Overtone: Creative Musical Experiences with Domain-Specific and Functional Languages},
  The First ACM SIGPLAN Workshop on Functional Art, Music, Modeling \& Design,
  Boston, Massachusetts, USA,
  ACM, pp. 35-46,
  (2013)

\bibitem{AOB14}
  Aaron, S., Orchard, D., and Blackwell, A.F.,
  \emph{Temporal Semanics for a Living Coding Language},
  Proceedings of the 2nd ACM SIGPLAN International Workshop on Functional Art, Music, Modeling \& Design,
  Sweden, ACM, pp. 37-47,
  (2014)

\bibitem{BAD14}
  Blackwell, A.F., Aaron, S., and Drury, R., 
  \emph{Exploring Creative Learning for the Internet of Things Era},
  In B. du Boulay and J. Good (Eds) Proceedings of the Psychology of Programming Interest Group Annual Conference, 
  pp. 147-158,
  (PPIG 2014)

\bibitem{BB92}
  Berry, G., and Boudol, G.,
  \emph{The chemical abstract machine},
  TCS, 96 pp.217–248, 
  (1992)

\bibitem{BC05}
  Blackwell, A.F., and Collins, N.,
  \emph{The Programming Language as a Musical Instrument},
  In Proceedings of the Psychology of Programming Interest Group Annual Conference,
  pp. 120-130,
  (PPIG 2005)

\bibitem{BMNR14}
  Blackwell, A., McLean, A., Noble, J., and Rohrhuber, J.,
  \emph{Collaboration and Learning Through Live Coding},
  Dagsthul Seminar, Dagstuhl Reports 3,
  no. 9, pp. 130-168,
  (2014)

\bibitem{Ch12}
  Church, L., Rothwell, N., Downie, M., deLahunta, S., and Blackwell, A.F.,
  \emph{Sketching by Programming in the Choreohraphic Language Agent},
  In Proceedings of the Psychology of Programming Interest Group Annual Conference,
  pp. 163-174,
  (PPIG 2012)

\bibitem{CCPY15}
  Coppo, M., Dezani-Ciancaglini, M., Padovani, L., and Yoshida, N.,
  \emph{A Gentle Introduction to Multiparty Asynchronous Session Types},
  SFM 2015, LNCS 9104, pp. 146-178,
  (2015)

\bibitem{DfEO13}
  Department for Education and Ofsted,
  \emph{ICT in schools: 2008 to 2011},
  Piccadilly Gate,
  Manchester,
  110134,
  (2013)

\bibitem{DfE13}
  Department for Education,
  \emph{National curriculum in England: computing programmes of study (key stages 1 - 4)},
  (2013)

\bibitem{HJ94}
  Hansson, H., and Jonsson, B.,
  \emph{A Logic for Reasoning About Time and Reliability},
  Formal Aspects of Computing 9,
  no 5., pp. 512-535,
  (1994)

\bibitem{HMBCY11}
  Honda, K., Mukhamedov, A., Brown, G., Chen, T., and Yoshida, N.,
  \emph{Scribbling Interactions with a Formal Foundation},
  In 7th International Conference on Distributed Computing and Internet Technology,
  p. 55-75,
  (ICDCIT 2011)

\bibitem{HVM98}
  Honda, K., Vasconcelos, V.T., and Kubo, M.,
  \emph{Language Primitives and Type Disciplines for Structured Communication-Based Programming},
  ESOP'98, LNCS 1381,
  pp. 22-38,
  (1998)

\bibitem{HYC08}
  Honda, K., Yoshida, N., and Carbone, M.,
  \emph{Multiparty Asynchronous Session Types},
  POPL'08, San Francisco, Calfornia, USA,
  pp. 273-284,
  (2008)

\bibitem{IG13}
  The IEEE and The Open Group,
  \emph{Sleep - The Open Group Base Specifications Issue 7, 2013},
  http://pubs.opengroup.org/onlinepubs/9699919799/functions/sleep.html,
  Retrieved 15 May,
  (2014)

\bibitem{M92}
   Milner, R.,
   \emph{Functions as processes},
   MSCS, 2(2) pp.119–141,
   (1992)

\bibitem{McD07}
  McDirmid, S.,
  \emph{Living it Up with a Live Programming Language},
  Proceedings of the 22Nd Annual ACM SIGPLAN Conference on Object-oriented Programming Systems and Applications,
  New York, USA,
  ACM, pp. 623-638
  (2007)

\bibitem{McL13}
  McLean, A.,
  \emph{The Textual X},
  Proceedings of xCoAx2013: Computation Communication Aesthetics and X,
  pp. 81-88,
  (2013)

\bibitem{ME14}
  McDirmid, S., and Edwards, J.,
  \emph{Programming with Managed Time},
  Tech. Report, Microsoft,
  (2014)

\bibitem{MY15}
  Mostrous, D., Yoshida, N., 
  \emph{Session typing and asynchronous subtyping for the higher-order pi-calculus}, 
  INFORMATION AND COMPUTATION, Vol: 241, 
  Pages: 227-263,
  (2015)

\bibitem{MYH09}
  Mostrous, D., Yoshida, N., Honda, K., 
  \emph{Global principal typing in partially
  commutative asynchronous sessions},
  in: ESOP’09, volume 5502 of
  LNCS, 
  Springer-Verlag, 
  pp. 316–332,
  (2009)

\bibitem{NY14}
  Ng, N., and Yoshida, N.,
  \emph{Pabble: Parameterised Scribble for Parallel Programming},
  In 22nd Euromicro International Conference on Parallel, Distributed and Network-Based Processing, 
  p. 707 - 714,
  (PDP 2014)

\bibitem{LDW87}
  Lee, I., Davidson, S., and Wolfe, V.,
  \emph{Motivating Time as a First Class Entity},
  Technical Reports (CIS),
  pp. 288,
  (1987)

\bibitem{mine1}
  Short, D.,
  \emph{Teaching Scientic Conceptsusing a Virtual World - Minecraft},
  Teaching Science,
  Volume 58, No. 3,
  pp. 55-58,
  (2012)

\bibitem{mine2}
  Smeaton, D.,
  \emph{Minecraft As A Teaching Tool - A Statistical Study of Teachers' Experience Using Minecraft In The Classroom},
  7267EDN Research Methods in Education,
  Griffith University, Austrailia,
  (2012)

\bibitem{SG10}
  Sorensen, A., and Gardner, H.,
  \emph{Programming with Time: Cyber-Physical Programming with Impromptu},
  ACM Sigplan Notcies 45,
  no. 10, 822-834,
  (2010)

\bibitem{T95}
  Thomasian, A.,
  \emph{Two-phase Locking Performance and Its Thrashing Behaviour},
  Performance of Concurrency Control Mechanisms in Centralized Database Systems
  Prentice-Hall, Inc.,
  Upper Saddle River, NJ, USA,
  pp. 166-214,
  (1995)

\bibitem{W10}
  Woolford, K., Blackwell, A.F., Norman, S.J., and Chevalier, C.,
  \emph{Crafting a Critical Technical Practice},
  Leonardo 43(2),
  202-203,
  (2010)

\bibitem{WC03}
  Wang, G., and Cook, P.R.,
  \emph{ChucK: A Concurrent, On-The-Fly Audo Programming Language},
  International Computer Music Conference,
  pp. 1-8,
  (2003)

\bibitem{Wing06}
  Wing, J.M.,
  \emph{Computational Thinking},
  Communication of the ACM,
  Vol. 49, pp. 33-35,
  (2006)

\bibitem{YT87}
  Yonezawa, A., and Tokoro, M.,
  \emph{Object-Oriented Concurrent Programming},
  MIT Press,
  (1987)

\end{thebibliography}

\end{document}